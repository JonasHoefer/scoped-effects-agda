\begin{code}[hide]
{-# OPTIONS --overlapping-instances #-}

module ScopedAlgebras where

open import Size using (Size; ↑_)
open import Function using (id; _∘_; _$_; const; case_of_)
open import Level using (Level)

open import Category.Monad using (RawMonad)
open        RawMonad ⦃...⦄ renaming (_⊛_ to _<*>_)

open import Data.Bool using (Bool; true; false; if_then_else_)
open import Data.Product using (Σ-syntax; _,_) renaming (proj₁ to π₁; proj₂ to π₂)
open import Data.Maybe using (Maybe; just; nothing)
open import Data.Nat using (ℕ; suc; _+_)
open import Data.Nat.Properties using (<-strictTotalOrder)
open import Data.Product using (_×_; _,_)
open import Data.Product.Relation.Binary.Lex.Strict using (×-strictTotalOrder)
open import Data.Sum using (_⊎_; inj₁; inj₂; [_,_])
open import Data.Empty using (⊥)
open import Data.List using (List; _∷_; []; _++_; foldr) renaming (map to mapᴸ)
-- open import Data.List.Relation.Unary.Any using (Any; here; there) renaming (map to map∈)
open import Data.Unit using (⊤; tt)

order = ×-strictTotalOrder (×-strictTotalOrder <-strictTotalOrder <-strictTotalOrder) <-strictTotalOrder
open import Data.Tree.AVL.Map order using (Map; empty; insert; lookup)

import Relation.Binary.PropositionalEquality as Eq
open   Eq using (_≡_; refl; cong; sym; trans)
open   Eq.≡-Reasoning using (_≡⟨⟩_; begin_; _∎; step-≡)

postulate
  extensionality : ∀ {ℓ ℓ′ : Level} {A : Set ℓ} {B : A → Set ℓ′} {f g : (x : A) → B x}
      → (∀ (x : A) → f x ≡ g x) → f ≡ g
\end{code}

Due to the deep-rooted problem with the higher order approach from Chapter
\ref{chapter:higher-order} it seemed reasonable to search for another
formulation of scoped effects, that does not rely on existential types.
\textcite{DBLP:conf/lics/PirogSWJ18} present a novel formalization for
scoped operations and their algebras, that fulfils this requirement.
In their paper they do not describe the combination of multiple effects, but due
to the different structure the approach seemed worth exploring nonetheless. % :/

This chapter transfers the basic implementation Haskell by
\textcite{DBLP:conf/lics/PirogSWJ18} to Agda.
To implement scoped algebras in an obviously terminating way, we derive a
different implementation of the \AgdaFunction{fold} by
\textcite{DBLP:conf/lics/PirogSWJ18}, based on the work by
\textcite{DBLP:journals/corr/abs-1806-05230}.
To test our implementation, we implement the nondeterminism example by
\textcite{DBLP:conf/lics/PirogSWJ18}, exceptions, state and sharing as defined
by \textcite{bunkenburg2019modeling}.
Furthermore, we present a new implementation for naive modularisation of
handlers.


\section{The Monad $E$}

\textcite{DBLP:conf/lics/PirogSWJ18} describe a monad, which is suited for
modelling operations with scopes.
The monad is the result of rewriting a slightly modified version of the monad
from ``Effect Handlers in Scope'', i.e. the monad used in Chapter
\ref{chapter:higher-order}.

In the following we describe the original and the resulting monad in Agda.
\textcite{DBLP:conf/lics/PirogSWJ18} separate the signature into one describing
effects with scopes $\Gamma$ and one describing the algebraic effects $\Sigma$.
Limiting our self to just strictly positive functors we can define the following
data type in Agda.

\begin{code}[hide]
record Container : Set₁ where
  constructor _▷_
  field
    Shape : Set
    Pos : Shape → Set
open Container public

_↝_ : Set → Set → Container
S ↝ P = S ▷ const P

⟦_⟧ : ∀ {ℓ} → Container → Set ℓ → Set ℓ
⟦ S ▷ P ⟧ A = Σ[ s ∈ S ] (P s → A)

-- TODO: EXPLAIN!
map : {A B : Set} {C : Container} → (A → B) → ⟦ C ⟧ A → ⟦ C ⟧ B
map f c = π₁ c , f ∘ π₂ c
{-# INLINE map #-}

_⊕_ : Container → Container → Container
(S₁ ▷ P₁) ⊕ (S₂ ▷ P₂) = (S₁ ⊎ S₂) ▷ [ P₁ , P₂ ]

Void : Container
Void = ⊥ ▷ λ()

sum : List Container → Container
sum = foldr _⊕_ Void

\end{code}
\begin{code}
data ProgE (Σ Γ : Container) (A : Set) : Set₁ where
  var  : A                                                    → ProgE Σ Γ A
  op   : ⟦ Σ ⟧ (ProgE Σ Γ A)                                  → ProgE Σ Γ A
  scp  : {X : Set} → ⟦ Γ ⟧ (ProgE Σ Γ X) → (X → ProgE Σ Γ A)  → ProgE Σ Γ A
\end{code}
This data type looks appropriate considering our earlier definition in Chapter
\ref{chapter:higher-order}.
An element of the monad is one of the following three.
A value of type $A$, construed using \AgdaInductiveConstructor{var}.
An operation without scopes, that is the functor $\Sigma$ applied to the monad
itself.
Notice that this case corresponds to the \AgdaInductiveConstructor{impure}
constructor in the free monad.
Or lastly, an operation with a local scope, represented by the second signature
$\Gamma$.
Operations with a local scope always quantifies over some type $X$, which is the
type for the intermediate result.
Furthermore, they provided a continuation, which transforms values of the
internal result type to the result type of the whole program.
The operations with scopes from Chapter \ref{chapter:higher-order} followed
exactly this pattern.
The shape of the by bi-container stored the intermediate result type and the two
position functions provided continuations resulting in programs with value type
\AgdaArgument{A} or \AgdaArgument{X}.
%\AgdaField{Ctx} and second position function were used to provide computations
%returning the result type and the second position function provided a
%continuation of the operation. % :/

Notice that this definition also requires $X$ to be a smaller type than
\AgdaDatatype{ProgE}, i.e. it is not sufficient to model deep effects as
explained in Chapter \ref{chapter:higher-order}.
\textcite{DBLP:conf/lics/PirogSWJ18} also note size issues with this definition.
\textcite{DBLP:conf/lics/PirogSWJ18} rewrite this monad using a left Kan
extension and derive the following equivalent monad without the existential
type.

\begin{code}
data ProgE′ (Σ Γ : Container) (A : Set) : Set where
  var  :  A                                   → ProgE′ Σ Γ A
  op   :  ⟦ Σ  ⟧ (ProgE′ Σ Γ A)               → ProgE′ Σ Γ A
  scp  :  ⟦ Γ  ⟧ (ProgE′ Σ Γ (ProgE′ Σ Γ A))  → ProgE′ Σ Γ A
\end{code}
An equivalent monad without the existential type is a promising candidate for
modelling deep effects in Agda, because it avoids the size issues.
Compared to the higher order approach, by using this monad we lose explicit
control over the parts of the program, which is important for potential
optimizations like memorization.

The monad models scopes using the double \AgdaDatatype{ProgE′} layer.
The outer layer corresponds to the part of program in scope.
By evaluating it (i.e. the part of the program in scope) one obtains a
computation, the rest of the program producing a result of type $A$.
This resulting program corresponds to the continuation in the higher order
approach, which is a function mapping from the result of the scoped program
to the rest of the whole program.
In their paper, \textcite{DBLP:conf/lics/PirogSWJ18} develop a theory of
\textit{scoped algebras} for a second equivalent monad and us them to define a
sensible evaluation for this monad.
In Section \ref{scoped-algebra:monad}, we define a more practical version of the
monad and translate part their practical results form Haskell to Agda. 


\section{The \texttt{Prog} Monad}
\label{scoped-algebra:monad}

In this section we define the central monad for this approach.
Because effects are split into signatures with and without scopes and most
computational effect have both, we define effects as pairs of signatures and
therefore \AgdaDatatype{Container}s.

\begin{code}
record Effect : Set₁ where
  field Ops Scps : Container
open Effect public
\end{code}
The functions \AgdaFunction{ops} and \AgdaFunction{scps} combine the
corresponding signatures of a \AgdaDatatype{List} of \AgdaDatatype{Effect}s.
\AgdaFunction{mapᴸ} corresponds to the Haskell function \texttt{map} for lists.

\begin{code}
ops scps : List Effect → Container
ops   = sum ∘ mapᴸ Ops
scps  = sum ∘ mapᴸ Scps
\end{code}
Using these helper functions we define a modular version of the monad by
\textcite{DBLP:conf/lics/PirogSWJ18}.
It is equivalent to the fixpoint and the Haskell definition from the paper
(assuming that signatures are given by functors which can be represented as
container). % needed? see last section

\begin{code}
data Prog (effs : List Effect) (A : Set) : Set where
  var  :  A                                          → Prog effs A
  op   :  ⟦ ops   effs  ⟧ (Prog effs A)              → Prog effs A
  scp  :  ⟦ scps  effs  ⟧ (Prog effs (Prog effs A))  → Prog effs A
\end{code}
\textcite{DBLP:conf/lics/PirogSWJ18} define a new kind of algebra for operations
with scopes, i.e. a recursion scheme for the above monad.
To implement their recursion scheme we have to define \AgdaFunction{<\$>} for 
\AgdaDatatype{Prog}\AgdaSpace{}\AgdaArgument{effs}\AgdaSpace{}\AgdaArgument{A}.
Notice that in the \AgdaInductiveConstructor{scp} constructor the type variable
\AgdaArgument{A} is instantiated with the type 
\AgdaDatatype{Prog}\AgdaSpace{}\AgdaArgument{effs}\AgdaSpace{}\AgdaArgument{A}
itself.
Such a data type is called a truly nested or non-regular data
type~\cite{DBLP:conf/mpc/BirdM98}.

Defining recursive functions for these data types is more complicated.
The ``direct'' implementations of \AgdaFunction{<\$>} and \AgdaFunction{>>=} for
\AgdaDatatype{Prog}\AgdaSpace{}\AgdaArgument{effs}\AgdaSpace{}\AgdaArgument{A}
do not obviously terminate.
Augmenting
\AgdaDatatype{Prog}\AgdaSpace{}\AgdaArgument{effs}\AgdaSpace{}\AgdaArgument{A}
with size annotations leads to problems while
proving the monad laws\footnote{As in Chapter \ref{chapter:higher-order} the
  monad laws only hold for size $\infty$, but working with size $\infty$ in
  proofs leads to termination problems.
  Repeated occurrences of the \AgdaInductiveConstructor{scp} constructor can
  produce an arbitrary number of
  \AgdaDatatype{Prog}\AgdaSpace{}\AgdaArgument{effs} layers.
  This forces us to use our induction hypothesis on not obviously smaller
  terms.}.
Instead, we define a generic \AgdaFunction{fold}, which we reuse for all
functions traversing values of type
\AgdaDatatype{Prog}\AgdaSpace{}\AgdaArgument{effs}\AgdaSpace{}\AgdaArgument{A}.


\subsection{Folds for Nested Data Types}
\label{scoped-algebra:fold}

Defining recursion schemes for complex recursive data types is a common problem.
\textcite{DBLP:journals/corr/abs-1806-05230} demonstrate a general construction
for \texttt{fold}s for nested data types in Agda.
Following their construction and adapting it to arbitrary branching trees leads
to a sufficiently strong \texttt{fold} to define all important functions for
\AgdaDatatype{Prog}\AgdaSpace{}\AgdaArgument{effs}\AgdaSpace{}\AgdaArgument{A}.

\paragraph{Index Type} The first part of the construction by
\textcite{DBLP:journals/corr/abs-1806-05230} is to define the correct index type
for the data structure.
The index type describes the recursive structure of the data type i.e. how the
type variables at each level are instantiated.
For complex mutual recursive data types the index type takes the form of a tree
with potentially different branch and leaf constructors.
For \AgdaDatatype{Prog}\AgdaSpace{}\AgdaArgument{effs}
the one type variable is either instantiated with the value type $A$
(\AgdaInductiveConstructor{pure}) or some number of
\AgdaDatatype{Prog}\AgdaSpace{}\AgdaArgument{effs} layers applied to
the current type variable (\AgdaInductiveConstructor{op},
\AgdaInductiveConstructor{scp}).
Our index type therefore just counts the number of
\AgdaDatatype{Prog}\AgdaSpace{}\AgdaArgument{effs} layers.
Natural numbers are sufficient.

Afterwards, \textcite{DBLP:journals/corr/abs-1806-05230} define a type level
function, that translates a value of the index type to the type it describes.
For
\AgdaDatatype{Prog}\AgdaSpace{}\AgdaArgument{effs}\AgdaSpace{}\AgdaArgument{A}
this function just applies \AgdaDatatype{Prog}\AgdaSpace{}\AgdaArgument{effs}
$n$ times to \AgdaArgument{A}.
The operator \AgdaFunction{\_\textasciicircum\_} applies a function $n$-times to
a given argument \AgdaArgument{x} i.e. it represents $n$-fold function
application, usually denoted $f^n(x)$.

\begin{code}[hide]
infixl 10 _^_
variable
  A B E S : Set
  effs : List Effect
\end{code}
\begin{code}
_^_ : ∀ {ℓ} {C : Set ℓ} → (C → C) → ℕ → C → C
(f ^ 0)      x = x
(f ^ suc n)  x = f ((f ^ n) x)
\end{code}

\paragraph{Recursion Scheme} Next we define \AgdaFunction{fold} for
\AgdaDatatype{Prog}\AgdaSpace{}\AgdaArgument{effs}\AgdaSpace{}\AgdaArgument{A}.
The fold produces value of an arbitrary \AgdaDatatype{ℕ} indexed type.
The type family $P$ determines the result type at each index.
The \AgdaFunction{fold} produces a value of type $P\;n$ for a given value of
type
$($\AgdaDatatype{Prog}\AgdaSpace{}\AgdaArgument{effs}\AgdaSpace{}\AgdaFunction{\textasciicircum}\AgdaSpace{}\AgdaArgument{n}$)$\AgdaSpace{}\AgdaArgument{A}.
Furthermore, the \texttt{fold} takes functions for processing substructures.
As usual, the arguments of these function correspond to those of the
constructors they process with recursive occurrences of the type itself already
processed.
In contrast to a normal recursion scheme, these recursive occurrences respect
the index.
For example, the function processing values constructed using
\AgdaInductiveConstructor{scp} takes an argument of type $P\;(2 + n)$, because
the constructor takes a value of type 
\AgdaDatatype{Prog}\AgdaSpace{}\AgdaArgument{effs}\AgdaSpace{}(\AgdaDatatype{Prog}\AgdaSpace{}\AgdaArgument{effs}\AgdaSpace{}\AgdaArgument{A}).
The cases are obtained by pattern matching on the index type and the value of
type
$($\AgdaDatatype{Prog}\AgdaSpace{}\AgdaArgument{effs}\AgdaSpace{}\AgdaFunction{\textasciicircum}\AgdaSpace{}\AgdaArgument{n}$)$\AgdaSpace{}\AgdaArgument{A}
i.e. the value of the type depending on the index.

\begin{code}
fold : (P : ℕ → Set) → ∀ n →
  (A                                           → P 0        )  →
  (∀ {n} → P n                                 → P (suc n)  )  →
  (∀ {n} → ⟦ ops   effs  ⟧  (P (suc n))        → P (suc n)  )  →
  (∀ {n} → ⟦ scps  effs  ⟧  (P (suc (suc n)))  → P (suc n)  )  →
  (Prog effs ^ n) A → P n
fold P 0        a v o s x         = a  x
fold P (suc n)  a v o s (var  x)  = v  (       fold P n              a v o s   x)
fold P (suc n)  a v o s (op   x)  = o  (map (  fold P (suc n)        a v o s)  x)
fold P (suc n)  a v o s (scp  x)  = s  (map (  fold P (suc (suc n))  a v o s)  x)
\end{code}
Notices that for the \AgdaInductiveConstructor{op} and
\AgdaInductiveConstructor{spc} constructors the recursive occurrences and
potential additional values are determined by the \AgdaDatatype{Container}.
Therefore, the functions processes arbitrary \AgdaDatatype{Container} extensions,
which contain solutions for the corresponding $P\;n$.
When implementing those two cases we cannot apply \AgdaFunction{fold} directly
to recursive occurrences, because those depend on the choice of
\AgdaDatatype{Container}.
Because container extensions are functors, we \AgdaFunction{map} over the given
value.

The \AgdaFunction{fold}, based on the approach by
\textcite{DBLP:journals/corr/abs-1806-05230}, is similar to the one by
\textcite{DBLP:conf/lics/PirogSWJ18}.
They derive their \AgdaFunction{fold} by constructing algebras for a
second monad.
Then \citetitle{DBLP:conf/lics/PirogSWJ18} show that the two monads are
equivalent and transfer the \AgdaFunction{fold} from the
second monad to the one used in their Haskell implementation.
Following \textcite{DBLP:journals/corr/abs-1806-05230} we derived essentially the
same \AgdaFunction{fold} and implemented it directly, without using
\AgdaFunction{>>=}.

\paragraph{Example}
Using \AgdaFunction{fold} we can implement \AgdaFunction{>>=} for
\AgdaDatatype{Prog}\AgdaSpace{}\AgdaArgument{effs}\AgdaSpace{}\AgdaArgument{A}.
The continuations \AgdaArgument{k} passed to \AgdaFunction{>>=} should just
affect the innermost layer.
Intuitively, when using bind a value constructed with
\AgdaInductiveConstructor{scp} we traverse the outer layer and call bind on the
inner one recursively.
To implement \AgdaFunction{>>=} we fold over the given program.
We could fix the arbitrary $n$ to be $1$, but defining a version generic in $n$
is useful for later proofs.
The \AgdaDatatype{ℕ} indexed type \AgdaDatatype{bind-P} defines the type for the
intermediate result for each layer.

\begin{code}
bind-P : ∀ A B effs → ℕ → Set
bind-P A B effs 0       = A
bind-P A B effs (suc n) = Prog effs ^ (suc n) $ B
\end{code}
\AgdaFunction{>>=} corresponds to variable substitution i.e. it replaces the
\AgdaInductiveConstructor{var} leafs with subtrees generated from their stored
values.
Calling $k$ on a program with zero layers (a value of type \AgdaArgument{A})
would produce a program with one layer.
Therefore, \bind{} can only be called on programs with at least one layer.
The lowest layer is extended with the result of \AgdaArgument{k}.
For all numbers greater than zero we produce a program with the same structure
but a different value type.
For layer zero (i.e. values) we could either produce a value of type 
\AgdaDatatype{Prog}\AgdaSpace{}\AgdaArgument{effs}\AgdaSpace{}\AgdaArgument{B}
or leave the value of type \AgdaArgument{A} unchanged.
In both cases we have to handle the lowest \AgdaInductiveConstructor{var}
constructors differently.

\begin{code}
bind : ∀ n → (Prog effs ^ suc n) A → (A → Prog effs B) →
  (Prog effs ^ suc n) B
bind {effs} {A} {B} n ma k = fold (bind-P A B effs) (suc n) id (λ where
    {0}      x → k x
    {suc n}  x → var x
  ) op scp ma
\end{code}
We left the values unchanged, therefore the \AgdaInductiveConstructor{var}
constructors at layer zero have to produce value of type
\AgdaDatatype{Prog}\AgdaSpace{}\AgdaArgument{effs}\AgdaSpace{}\AgdaArgument{B}
by calling \AgdaArgument{k}.
In all other cases we replace each constructor with itself to leave these parts
unchanged.

Using \AgdaFunction{>>=} for a program with one layer and
\AgdaInductiveConstructor{var} as \AgdaFunction{return} we can define a monad
instance for \AgdaDatatype{Prog}\AgdaSpace{}\AgdaArgument{effs}.
In contrast to Haskell this automatically defines a functor and an applicative
instance, because these can be defined in terms of \AgdaFunction{>>=} and
\AgdaFunction{return}\footnote{We automatically obtain the functions
  \AgdaFunction{<\$>}, \AgdaFunction{<*>}, \AgdaFunction{pure} and
  \AgdaFunction{>>}. }.
\begin{code}
instance
  Prog-RawMonad : RawMonad (Prog effs)
  Prog-RawMonad = record { return = var ; _>>=_ = bind 0 }
\end{code}
In contrast to Chapter \ref{chapter:first-order} this is not a problem, because
we do not have the meta problem of preserving sizes.
The record is part of the Agda standard library.
It is named \AgdaDatatype{RawMonad}, because it does not enforce the monad laws.
We prove the laws in Section \ref{scoped-algebra:monad-laws} using
a technique from Section \ref{scoped-algebra:ind}.


\subsection{Induction Schemes for Nested Data Types}
\label{scoped-algebra:ind}

Following the examples by \textcite{DBLP:journals/corr/abs-1806-05230} we
generalize the \AgdaFunction{fold} from Section \ref{scoped-algebra:fold} to a
dependently typed version, an induction principle.
In the induction principle the proposition \AgdaDatatype{P} is generalized to a
dependent type i.e. a relation on values of type 
\AgdaDatatype{Prog}\AgdaSpace{}\AgdaArgument{effs}\AgdaSpace{}\AgdaArgument{A}.
Therefore, the induction principle allows for proofs of these predicates by
induction without the use of explicit recursion.

The types for the four functions, corresponding to the three constructors and
the base case, also have to be generalized.

\begin{code}
ind : (P : (n : ℕ) → (Prog effs ^ n) A → Set) → ∀ n →
\end{code}
In the base case the given value of type \AgdaArgument{A} is the handled value.
We bind it to \AgdaArgument{x} and pass it to the proposition.
\begin{code}
  (                (x : A)                                        → P 0 x                     )  →
\end{code}
The \AgdaInductiveConstructor{var} case can be handled similar to the examples by
\textcite{DBLP:journals/corr/abs-1806-05230}.
We assume values for all constructor arguments, as well as proofs for the
smaller cases.
For \AgdaInductiveConstructor{var} an additional, hidden argument
\AgdaArgument{x} is introduced.
\AgdaArgument{x} represents a recursive occurrence, therefore it is part of the
type of the proof for the smaller case $P\;n\;x$.
\AgdaArgument{x} is used to describe the currently handled value
\AgdaInductiveConstructor{var}\AgdaSpace{}\AgdaArgument{x}.
Therefore, \AgdaInductiveConstructor{var}\AgdaSpace{}\AgdaArgument{x} is part of
the functions result type.
Under the Curry Howard corresponds this function (ignoring the index type) can
be read as the proposition
$\forall x. P(x) \rightarrow P(\AgdaInductiveConstructor{var}\;x)$
i.e. the induction step for the constructor \AgdaInductiveConstructor{var}.
\begin{AgdaAlign}
\begin{code}
  (∀ {n x} → P n x → P (suc n) (var x))  →
\end{code}
Similar to \AgdaFunction{fold} dealing with \AgdaInductiveConstructor{op} and
\AgdaInductiveConstructor{scp} is more complicated, because they represent
arbitrarily branching nodes.
Remember that a container's position function maps from a type of positions,
which depends on the values shape, to the contained values.
Hence, assuming the containers values corresponds to assuming a position
function.
Analogues to the \AgdaFunction{fold}, the result for the recursively handled value
is a function from positions (for the assumed shape) to proofs for the
proposition for the there contained values.
The value for each position is obtained using the assumed position function
\AgdaArgument{κ}.
The currently handled value is constructed using \AgdaArgument{s} and  
\AgdaArgument{κ}.
\begin{code}
  (∀ {n} s {κ}  →  ((p : Pos (ops   effs) s) → P (1 + n)  (κ p))  → P (suc n) (op   (s , κ))  )  →
  (∀ {n} s {κ}  →  ((p : Pos (scps  effs) s) → P (2 + n)  (κ p))  → P (suc n) (scp  (s , κ))  )  →
  (x : (Prog effs ^ n) A) → P n x
\end{code}
The actual implementation of the induction principle is straight forward and
similar to \AgdaFunction{fold}.
The composition with the position function corresponds to the call to
\AgdaFunction{map}.
\begin{code}
ind P 0       a v o s x               = a x
ind P (suc n) a v o s (var x)         = v    (ind P n              a v o s x)
ind P (suc n) a v o s (op   (c , κ))  = o c  (ind P (suc n)        a v o s ∘ κ)
ind P (suc n) a v o s (scp  (c , κ))  = s c  (ind P (suc (suc n))  a v o s ∘ κ)
\end{code}
\end{AgdaAlign}


\subsection{Proving the Monad Laws}
\label{scoped-algebra:monad-laws}

To demonstrate the use of the induction principle from Section
\ref{scoped-algebra:ind} and to verify that 
\AgdaDatatype{Prog}\AgdaSpace{}\AgdaArgument{effs}\AgdaSpace{}\AgdaArgument{A}
is a monad we prove the monad laws.

\paragraph{Left Identity}
First we prove the left identity law.
Because the value passed \AgdaFunction{>>=} is constructed using
\AgdaFunction{return} i.e. \AgdaInductiveConstructor{var}, both sides of the
equality evaluate to \AgdaArgument{k}\AgdaSpace{}\AgdaArgument{a}.

\begin{code}
bind-identˡ : ∀ {A B : Set} {a} {k : A → Prog effs B} →
  (return a >>= k) ≡ k a
bind-identˡ = refl
\end{code}

\paragraph{Right Identity}
Next we prove the right identity law.
The value passed to \AgdaFunction{>>=} can be an
arbitrarily complex program.
Therefore, the proposition is not ``obvious'' as it was the case with
\AgdaFunction{bind-identˡ}.

\begin{AgdaAlign}
\begin{code}
bind-identʳ : {A : Set} (ma : Prog effs A) → (ma >>= return) ≡ ma
\end{code}
\AgdaFunction{>>=} is recursive in its first argument, therefore we prove the
proposition by induction on \AgdaArgument{ma}.
To use the induction principle we have to define a proposition for every layer.
Since \AgdaFunction{>>=} does not change values, we simply produce a value of
type \AgdaDatatype{⊤} at layer zero.
For all other we prove the proposition for the appropriate
\AgdaFunction{bind} i.e. the one called by \AgdaFunction{fold} to handle a value
for the given \AgdaArgument{n}.

\begin{code}
bind-identʳ {effs} {A} = ind (λ{ 0 _ → ⊤ ; (suc n) p → bind n p var ≡ p })
\end{code}
We call the induction principle with the given value with one layer, therefore
the initial \AgdaArgument{n} is \AgdaNumber{1}.
The ``proof'' for values can be inferred, because for each value it's just a
value of the unit type.

\begin{code}
  1 _
\end{code}
To prove that the proposition holds for values constructed using
\AgdaInductiveConstructor{var} we case split on the number of layers
\AgdaArgument{n}.
For \AgdaNumber{0} we prove that
\AgdaInductiveConstructor{var}\AgdaSpace{}\AgdaArgument{x} is equal to itself,
because \AgdaFunction{>>=} leaves values unchanged.
This proof forms the basis for our induction.

For the second case we are given a program \AgdaArgument{x} with
\AgdaInductiveConstructor{suc}\AgdaSpace{}\AgdaArgument{n} layers and an
induction hypothesis \AgdaArgument{IH} that the proposition holds for a call to
\AgdaFunction{bind}\AgdaSpace{}\AgdaArgument{n}.
We have to prove that the proposition holds for
\AgdaFunction{bind}\AgdaSpace{}$($\AgdaInductiveConstructor{suc}\AgdaSpace\AgdaArgument{n}$)$
and the new program \AgdaInductiveConstructor{var}\AgdaSpace{}\AgdaArgument{x}.
By applying the \AgdaFunction{>>=} rule for \AgdaInductiveConstructor{var} (for
\AgdaArgument{n} greater than zero) we move \AgdaInductiveConstructor{var} to
the outside.
The new proposition is
\AgdaInductiveConstructor{var}\AgdaSpace{}$($\AgdaFunction{bind}\AgdaSpace{}\AgdaArgument{n}$)$\AgdaSpace{}\AgdaDatatype{≡}\AgdaSpace{}\AgdaInductiveConstructor{var}\AgdaSpace{}\AgdaArgument{x}.
This proposition is equal to the induction hypothesis with
\AgdaInductiveConstructor{var} applied on both sides.
Therefore, we prove it using congruence.

\begin{code}
  (λ{ {0} (tt) → refl ; {suc n} IH → cong var IH })
\end{code}
The proofs for the \AgdaInductiveConstructor{op} and
\AgdaInductiveConstructor{scp} are identical and similar to the
\AgdaInductiveConstructor{var} case. For each shape \AgdaArgument{s} we are
given a proof for each position of the shape.
We have to prove that the proposition holds for a new program constructed using
either \AgdaInductiveConstructor{op} or \AgdaInductiveConstructor{scp}.
For both constructors \AgdaFunction{bind}\AgdaSpace{}\AgdaArgument{n} calls
itself recursively on the all contained values i.e. the results of the position
function.
Using congruence we can simplify the equality to the equality of the position
functions.
By invoking the axiom of \AgdaFunction{extensionality} we prove the equality
point wise i.e. for each position.
This equality is exactly the one given by the induction hypothesis.

\begin{code}
  (λ s IH → cong (op   ∘ (s ,_)) (extensionality IH))
  (λ s IH → cong (scp  ∘ (s ,_)) (extensionality IH))
\end{code}
\end{AgdaAlign}

\paragraph{Associativity}
The proof for associativity follows the same pattern as the one for the right
identity.
\AgdaFunction{>>=} is defined by recursion on its first argument.
Therefore, we proof the proposition by induction on \AgdaArgument{ma}.
In all cases the left-hand side reduces to the induction hypothesis in the same
manner as before.
Therefore, these cases look identical.

\begin{code}
bind-assoc : ∀ {A B C}
  (f : A → Prog effs B) (g : B → Prog effs C) (ma : Prog effs A) →
  (ma >>= f >>= g) ≡ (ma >>= λ a → f a >>= g)
bind-assoc f g = ind
  (λ where
    0 p        → ⊤
    (suc n) p  → bind n (bind n p f) g ≡ bind n p λ a → bind 0 (f a) g
  ) 1 _
  (λ{ {0} _ → refl ; {suc n} IH → cong var IH})
  (λ s IH → cong (op   ∘ (s ,_)) (extensionality IH))
  (λ s IH → cong (scp  ∘ (s ,_)) (extensionality IH))
\end{code}

\paragraph{Functor and Applicative Laws}
The monad laws imply the functor and applicative laws.
We can proof them without using explicit induction by rewriting equations
involving \AgdaFunction{>>=} using the above laws.
These proofs are simpler than the ones explicitly involving the definition of
\AgdaFunction{>>=} using \AgdaFunction{fold}.
We can write them using the chain reasoning operators as described by
Norell~\cite{norell:thesis}.

We just prove a single law to demonstrate the general structure.
By defining the monad instance the applicative and functor operators are 
defined in terms of \AgdaFunction{>>=}.
Therefore, we replace \AgdaFunction{<\$>} with its definition.
By simplifying the term we can apply the right identity law for monads, which
yields the correct result.

\begin{code}
fmap-id : ∀ {effs} {A B : Set} → (ma : Prog effs A) → (id <$> ma) ≡ ma
fmap-id ma = begin
  (id <$> ma)                ≡⟨⟩ -- definition of <$>
  (ma >>= λ a → var (id a))  ≡⟨⟩ -- definition of id and η-conversion
  (ma >>= var)               ≡⟨ bind-identʳ ma ⟩
  ma                         ∎
\end{code}
Proofs for the other laws can be found in the repository.
They are used to prove properties of effect handlers.


\section{Combining Effects}

Similar to Chapter \ref{chapter:higher-order} we reuse parts of the
infrastructure from Chapter \ref{chapter:first-order}.
Each \AgdaDatatype{Effect} consists of a pair of \AgdaDatatype{Container}s, one
representing scoped and one representing algebraic effects.
Therefore, an \AgdaDatatype{Effect} stack now stores pairs of containers, not
containers directly.
The type \AgdaDatatype{\_∈\_} together with \AgdaFunction{inj} and
\AgdaFunction{prj} can be reused to model effect constraints.

\begin{code}[hide]
variable
  C  : Container
  Cs : List Container
  e : Effect
\end{code}

\begin{code}[hide]
infix 4 _∈_
data _∈_ {ℓ : Level} {A : Set ℓ} (x : A) : List A → Set ℓ where
  instance
    here   : ∀ {xs} → x ∈ x ∷ xs
    there  : ∀ {y xs} → ⦃ x ∈ xs ⦄ → x ∈ y ∷ xs

inj : C ∈ Cs → ⟦ C ⟧ A → ⟦ sum Cs ⟧ A
inj here          (s , κ) = inj₁ s , κ
inj (there ⦃ p ⦄) q       with inj p q
... | s , κ = inj₂ s , κ

prj : C ∈ Cs → ⟦ sum Cs ⟧ A → Maybe (⟦ C ⟧ A)
prj here           (inj₁ s , κ) = just (s , κ)
prj here           (inj₂ s , κ) = nothing
prj (there ⦃ p ⦄)  (inj₁ s , κ) = nothing
prj (there ⦃ p ⦄)  (inj₂ s , κ) = prj p (s , κ)
\end{code}
Due to the component wise combination of \AgdaDatatype{Effect}s, a proof that an
\AgdaDatatype{Effect} is an element of an effect stack implies that an
operation from one of the two signatures is part of the corresponding
signature of the combined effect.

\begin{code}
opsInj : e ∈ effs → Ops e ∈ mapᴸ Ops effs
opsInj here            = here
opsInj (there ⦃ p ⦄)   = there ⦃ opsInj p ⦄

scpsInj : e ∈ effs → Scps e ∈ mapᴸ Scps effs
scpsInj here           = here
scpsInj (there ⦃ p ⦄)  = there ⦃ scpsInj p ⦄
\end{code}
Using the above proofs we can implement smart constructors for scoped and
algebraic operations without requiring additional evidence.

\begin{code}
Op : ⦃ e ∈ effs ⦄ → ⟦ Ops e ⟧ (Prog effs A) → Prog effs A
Op ⦃ p ⦄ = op ∘ inj (opsInj p)

Scp : ⦃ e ∈ effs ⦄ → ⟦ Scps e ⟧ (Prog effs (Prog effs A)) → Prog effs A
Scp ⦃ p ⦄ = scp ∘ inj (scpsInj p)
\end{code}
To escape the monad after interpreting all effects we again add the
handler for the \AgdaDatatype{Void} effect.

\begin{code}
run : Prog [] A → A
run (var x) = x
\end{code}


\section{Nondeterministic Choice}

As first example, we implement the nondeterminism effect.
In Section \ref{scoped-algebra:nondet:scoped-algebra}, we implement the example
by \textcite{DBLP:conf/lics/PirogSWJ18} to demonstrate that the
\AgdaFunction{fold} definition from Section \ref{scoped-algebra:fold} is
sufficient.
Section \ref{scoped-algebra:nondet:modular} presents an idea for modular effects
and handlers.

\subsection{Implementation as Scoped Algebra}
\label{scoped-algebra:nondet:scoped-algebra}
The signature for the two algebraic operations is the same as before.
Furthermore, we define a signature for the scoped operation \AgdaFunction{once}.

\begin{code}[hide]
SID CID : Set
SID = ℕ × ℕ
CID = SID × ℕ
\end{code}
\begin{code}
data Choiceˢ  : Set where ⁇ˢ : (Maybe CID) → Choiceˢ ; failˢ : Choiceˢ
data Onceˢ    : Set where onceˢ : Onceˢ
\end{code}
\AgdaDatatype{Nondetᵖ} corresponds to the data type from the example by Piróg et
al.~\cite[sec.~6]{DBLP:conf/lics/PirogSWJ18}.
The effect has the unary, scoped operation \AgdaFunction{once} and the nullary
and binary, algebraic operations \AgdaFunction{⁇} and \AgdaFunction{fail}.

\begin{code}
Nondetᵖ : Effect
Ops   Nondetᵖ = Choiceˢ  ▷ λ{ (⁇ˢ _) → Bool ; failˢ → ⊥ }
Scps  Nondetᵖ = Onceˢ    ↝ ⊤
\end{code}
A scoped algebra for an \AgdaDatatype{ℕ} indexed carrier type \AgdaArgument{C}
consists of three operations.
\AgdaField{a} the \textit{algebra} for the algebraic operations.
\AgdaField{p} for \textit{promoting} the carrier type when entering a scope.
\AgdaField{d} for \textit{demoting} the carrier type when leaving a scope and
interpreting the scoped operation.
Figure \ref{scoped-algebra:fig:reduction} demonstrates how the handler
interprets \AgdaDatatype{Nondet} syntax using the scoped algebra.
The carrier type are $n$-fold lists.
The example is taken from \textcite{DBLP:conf/lics/PirogSWJ18}, but notated
using
\AgdaDatatype{Prog}\AgdaSpace{}\AgdaArgument{effs}\AgdaSpace{}\AgdaArgument{A}.
\textbf{var} nodes are handled using \AgdaField{p}.
The value is promoted by lifting it into a singleton list.
The algebraic operations \textbf{fail} and \textbf{?} are handled using
\AgdaField{a}.
Notice that in the definition of \AgdaFunction{fold} they are guaranteed an index
$n\geqslant 1$ i.e. at least one list layer.
They are implement identical to earlier versions of \AgdaDatatype{Nondet}.
\textbf{?} concatenates the left and right list.
In the third step the lists are promoted again, because they enter the scope
created by \textbf{once}.
This is key, because for algebraic operations in the scope the multiple results
from outside the scope appear as a single value.
\textbf{once} is evaluated using \AgdaField{d}, removing the scope.
\textbf{once} keeps just the first result inside the scope i.e. the list
containing the results of the continuation applied to the result of
$\textbf{once}\;(\textbf{var}\;1\;?\;\textbf{var}\;3)\equiv [1]$.

\begin{figure}
  \begin{tikzpicture}[level distance=.7cm,sibling distance=1.3cm]
  \node[treenode] at (0,0) {\textbf{scp once}}
    child { 
      node[treenode] {\textbf{?}}
      child {
        node[treenode] {\textbf{var}}
        child[sibling distance=.6cm] {
          node[treenode] {\textbf{?}}
          child {
            node[treenode] {\textbf{var}}
            child { node[treenode] {$1$} }
          }
          child {
            node[treenode] {\textbf{var}}
            child { node[treenode] {$2$} }
          }
        }
      }
      child {
        node[treenode] {\textbf{var}}
        child[sibling distance=.6cm] {
          node[treenode] {\textbf{?}}
          child {
            node[treenode] {\textbf{var}}
            child { node[treenode] {$3$} }
          }
          child {
            node[treenode] {\textbf{var}}
            child { node[treenode] {$4$} }
          }
        }
      }
    }
  ;
  \node[treenode] at (1.5,0) {$\stackrel{p}{\leadsto{}}$};
  \node[treenode] at (3,0) {\textbf{scp once}}
    child { 
      node[treenode] {\textbf{?}}
      child {
        node[treenode] {\textbf{var}}
        child[sibling distance=.6cm] {
          node[treenode] {\textbf{?}}
          child { node[treenode] {$[1]$} }
          child { node[treenode] {$[2]$} }
        }
      }
      child {
        node[treenode] {\textbf{var}}
        child[sibling distance=.6cm] {
          node[treenode] {\textbf{?}}
          child { node[treenode] {$[3]$} }
          child { node[treenode] {$[4]$} }
        }
      }
    }
  ;
  \node[treenode] at (4.5,0) {$\stackrel{a}{\leadsto{}}$};
  \node[treenode] at (6,0) {\textbf{scp once}}
    child { 
      node[treenode] {\textbf{?}}
      child {
        node[treenode] {\textbf{var}}
        child[sibling distance=.6cm] {
          node[treenode] {$[1,2]$}
        }
      }
      child {
        node[treenode] {\textbf{var}}
        child[sibling distance=.6cm] {
          node[treenode] {$[3,4]$}
        }
      }
    }
  ;
  \node[treenode] at (7.25,0) {$\stackrel{p}{\leadsto{}}$};
  \node[treenode] at (8.5,0) {\textbf{scp once}}
    child { 
      node[treenode] {\textbf{?}}
      child {
        node[treenode] {$[[1,2]]$}
      }
      child {
        node[treenode] {$[[3,4]]$}
      }
    }
  ;
  \node[treenode] at (9.75,0) {$\stackrel{a}{\leadsto{}}$};
  \node[treenode] at (11,0) {\textbf{scp once}}
    child { 
      node[treenode] {$[[1,2],[3,4]]$}
    }
  ;
  \node[treenode] at (12.25,0) {$\stackrel{d}{\leadsto{}}$};
  \node[treenode] at (13,0) {$[1,2]$};
  \end{tikzpicture}
  \label{scoped-algebra:fig:reduction}
  \caption{Interpretation of
    $\textbf{once}\;(\textbf{var}\;1\;?\;\textbf{var}\;3)\bind \lambda
    x.\textbf{var}\;x\;?\;\textbf{var}\;(x+1)$}
\end{figure}

The indices for \AgdaField{d} and \AgdaField{a} are offset by one compared to
the definition by Piróg et al.
We do not define our carrier data type in two stages as Piróg et al. but as an
$n$-fold version of a simpler type.
This change allows a simpler implementation using the \AgdaFunction{fold} from
Section \ref{scoped-algebra:fold}, but should be insignificantly enough to
present the similarities between the two recursion schemes.

\begin{code}
record ScopedAlgebra (E : Effect) (C : ℕ → Set) : Set where
  constructor ⟨_,_,_⟩
  field
    p : ∀ {n} → C n                      → C (1 + n)
    d : ∀ {n} → ⟦ Scps  E ⟧ (C (2 + n))  → C (1 + n)
    a : ∀ {n} → ⟦ Ops   E ⟧ (C (1 + n))  → C (1 + n)
open ScopedAlgebra
\end{code}
Given a \AgdaDatatype{ScopedAlgebra} for an \AgdaDatatype{Effect} we
interpret its syntax.
\AgdaDatatype{foldᴾ} corresponds to a slightly modified version of the fold by
\textcite{DBLP:conf/lics/PirogSWJ18}.
Their function works for an arbitrary \AgdaArgument{n}, because it is defined
recursively. 
In this implementation the recursion is abstracted in \AgdaFunction{fold},
therefore this is not needed.
Furthermore, they have to inject values explicitly into the carrier type.
This happens implicitly due to the handling of the $0$-th layer.
Notice that the functions from the \AgdaDatatype{ScopedAlgebra} correspond
exactly to the ones expected by the \AgdaFunction{fold} from Section
\ref{scoped-algebra:fold}.
The values them self are not preprocessed and just get promoted, hence
\AgdaFunction{id} is passed as the first of the four arguments.

\begin{code}
foldᴾ : ∀ {C : Set → ℕ → Set} → ScopedAlgebra e (C A) → 
  Prog (e ∷ []) (C A 0) → C A 1
foldᴾ {C = C} ⟨ p , d , a ⟩ = fold (C _) 1 id p
  (λ{ (inj₁ s , κ) → a (s , κ) }) λ{ (inj₁ s , κ) → d (s , κ) }
\end{code}
Just porting the algebra for \AgdaDatatype{Nondet} by
\textcite{DBLP:conf/lics/PirogSWJ18} yields a
correct handler for the effect in isolation.
The carrier type are iterated \AgdaDatatype{List}s.
The implementation of the scoped algebra is straight forward.
The algebraic operations are handled as before and the promotion operation
corresponds to the earlier handling of values.
The demotion operation i.e. the handler for \AgdaFunction{once} has
to produce an $n$-fold list, given an $n+1$-fold list.
Before applying the handler, the programs in the \AgdaInductiveConstructor{var}
nodes are the results for all possible continuations for the program in scope.
The elements of the given list are the results of these programs.
By acting just on the given list of lists the handler can separate between the
results of the different branches in the scoped program.
\AgdaFunction{once} is implemented by taking the head of the list if possible
i.e. the results of the continuation applied to the first result of the program
in scope.

\begin{code}
NondetAlg : ScopedAlgebra Nondetᵖ (λ i → List ^ i $ A)
p NondetAlg              = _∷ []
a NondetAlg (⁇ˢ _  , κ)  = κ true ++ κ false
a NondetAlg (failˢ , κ)  = []
d NondetAlg (onceˢ , κ)  = case κ tt of λ where
  []       → []
  (x ∷ _)  → x

runNondetᴾ : Prog (Nondetᵖ ∷ []) A → List A
runNondetᴾ = foldᴾ {C = λ A i → List ^ i $ A} NondetAlg
\end{code}
The carrier type can be thought of as the context for the computation.
By having contexts of contexts it is possible to differentiate between the state
of the computation in scope and the whole computation.

Lastly we define a smart constructor for the \AgdaFunction{once} operation.
To capture the program \AgdaArgument{p} in scope, \AgdaFunction{pure} is mapped
over it.
The original program is now the outer
\AgdaDatatype{Prog}\AgdaSpace{}\AgdaArgument{effs} layer i.e. the first and
second layers of the first term in Figure \ref{scoped-algebra:fig:reduction}.
The original \AgdaInductiveConstructor{var}\AgdaSpace{}\AgdaArgument{x} nodes
are now
\AgdaInductiveConstructor{var}$($\AgdaInductiveConstructor{var}\AgdaSpace{}\AgdaArgument{x}$)$
nodes.
Since \AgdaFunction{>>=} just affects the lowest layer it won't change the
captured program and just act on its results.

\begin{code}
onceᵖ : ⦃ Nondetᵖ ∈ effs ⦄ → Prog effs A → Prog effs A
onceᵖ p = Scp (onceˢ , λ _ → pure <$> p)
\end{code}

\subsection{Implementation as Modular Handler}
\label{scoped-algebra:nondet:modular}

\textcite{DBLP:conf/lics/PirogSWJ18} just present handlers for single effects
and make some remarks regarding modularity in a theoretical context for their
equivalent monad.
We work in a more practical context, i.e. we are just concerned with a small
number of possible effects and can therefore impose more restrictions on
them.
This section presents a naive idea for adapting the handlers from the paper to
modular ones by following patterns from earlier chapters.
As before, we will be working with the \AgdaDatatype{Prog} monad, i.e. the one
used by \textcite{DBLP:conf/lics/PirogSWJ18} to implement effects, not the one
used to define scoped algebras.

As in the earlier chapters, when working with modular effects, the general
approach is to execute the handlers for each effect one after
another~\cite{DBLP:conf/haskell/SchrijversPWJ19}.
Each handler interprets the syntax for its effect and leaves the rest in place
(or in case of non-orthogonal effects manipulates syntax of certain other
effects).
The carrier type for these handlers consists of the usual carrier type for these
handlers, which is post-composed with the type of the program without the
interpreted syntax.
For example, a handler for exceptions would produce values of type
\AgdaArgument{E}\AgdaSpace{}\AgdaDatatype{⊎}\AgdaSpace{}\AgdaArgument{A},
therefore the modular handler produces value of type 
\AgdaDatatype{Prog}\AgdaSpace{}\AgdaArgument{effs}\AgdaSpace{}$($\AgdaArgument{E}\AgdaSpace{}\AgdaDatatype{⊎}\AgdaSpace{}\AgdaArgument{A}$)$.

In the context of scoped algebras, the \AgdaDatatype{ℕ} indexed carrier type is
usually the $n$-fold of some simpler type $C$.
It seems reasonable to choose an $n$-fold of
\AgdaDatatype{Prog}\AgdaSpace{}\AgdaArgument{effs}\AgdaSpace{}$\circ\; C$ as
carrier type, because the structure of $n$ layers of $C$s has to be preserved to
reuse the non-modular handler.
Simply lifting the $n$-fold carrier type in the monad is not sufficient.
The interleaved monad layers are needed to preserve non-interpreted syntax.

Consider the following implementation of a modular handler for nondeterminism.
This version of the nondeterminism effect does not support the
\AgdaFunction{once} operation, because the semantics of the above definition is
not the expected one when interacting with other effects.
Similar to earlier chapters we introduce \AgdaKeyword{pattern}s to simplify
the handler.

\begin{code}[hide]
data Tree (A : Set) : Set where
  branch : Maybe CID → (l r : Tree A) → Tree A
  leaf   : A → Tree A
  failed : Tree A

dfs : {A : Set} → Map Bool → Tree A → List A
dfs mem failed                   = []
dfs mem (leaf x)                 = x ∷ []
dfs mem (branch nothing    l r)  = dfs mem l ++ dfs mem r
dfs mem (branch (just id)  l r) with lookup id mem
... | nothing  = dfs (insert id true mem) l ++ dfs (insert id false mem) r
... | just d   = if d then dfs mem l else dfs mem r
\end{code}
\begin{code}
Nondet : Effect
Ops   Nondet = Choiceˢ ▷ λ{ (⁇ˢ _) → Bool ; failˢ → ⊥ }
Scps  Nondet = Void

pattern Other s κ     = (inj₂ s , κ)
pattern Choice cid κ  = (inj₁ (⁇ˢ cid) , κ)
pattern Fail          = (inj₁ failˢ , _)
\end{code}
The new handler has the expected signature and uses the carrier type described
above.

\begin{AgdaAlign}
\begin{code}
runNondet′ : Prog (Nondet ∷ effs) A → Prog effs (Tree A)
runNondet′ {effs} {A} = fold (λ i → (Prog effs ∘ Tree) ^ i $ A) 1 id
\end{code}
First we have to handle the \AgdaInductiveConstructor{var} case.
Values are now not only injected into the context, but also into the monad.

\begin{code}
  (pure ∘ leaf)
\end{code}
The algebraic operations are interpreted as before, except that all results are
lifted into the monad.
When an operation from a foreign effect is encountered, the syntax is just
reconstructed.
The outer container coproduct is removed by interpreting the
\AgdaDatatype{Nondet} syntax, therefore \AgdaArgument{s} already has the correct
type (note that \AgdaInductiveConstructor{Other} hides the
\AgdaInductiveConstructor{inj₂}).
\AgdaArgument{κ} produces the result for the positions of the given shape,
because the recursion is handled by \AgdaFunction{fold}.
Therefore, the implementation is the same as in Chapter \ref{chapter:first-order}
except that the recursion is hidden.

\begin{code}
  (λ where
    (Choice id κ)  → branch id <$> κ true <*> κ false
    Fail           → pure failed
    (Other s κ)    → op (s , κ)
\end{code}
The interesting case is the one for scoped operations of foreign effects.
Similar to the algebraic operations we have to reconstruct the scoped operation,
but the position function has the wrong type.
It produces values of type
\AgdaDatatype{Prog}\AgdaSpace{}\AgdaArgument{effs}\AgdaSpace{}\AgdaFunction{∘}\AgdaSpace{}\AgdaDatatype{Tree}\AgdaSpace{}\AgdaFunction{\textasciicircum}\AgdaSpace{}\AgdaInductiveConstructor{suc}\AgdaSpace{}\AgdaArgument{n}\AgdaSpace{}\AgdaFunction{\$}\AgdaSpace{}\AgdaArgument{A}
but expected are values of type
\AgdaDatatype{Prog}\AgdaSpace{}\AgdaArgument{effs}\AgdaSpace{}\AgdaFunction{\$}
\AgdaDatatype{Prog}\AgdaSpace{}\AgdaArgument{effs}\AgdaSpace{}\AgdaFunction{∘}\AgdaSpace{}\AgdaDatatype{Tree}\AgdaSpace{}\AgdaFunction{\textasciicircum}\AgdaSpace{}\AgdaArgument{n}\AgdaSpace{}\AgdaFunction{\$}\AgdaSpace{}\AgdaArgument{A}.

To remove the intermediate \AgdaDatatype{Tree} layer we map the function
\AgdaFunction{hdl} over the result of the continuation.
\AgdaFunction{hdl} traverses the outer tree and recombines the inner trees under
the monad i.e. by interleaving binds it orders the results.
On a term level, we are given a list of possible continuations, which we execute
on after another, appending their results.

Similar to the handler from Section \ref{higher-order:lifting}, this
implementation orders the results correctly, but not the operations of other
effects, which entails possible unexpected interaction semantics.

\begin{code}
  ) λ where
    (Other s κ)  → scp (s , λ p → hdl <$> κ p)
  where
    hdl : ∀ {A} → Tree (Prog effs (Tree A)) → Prog effs (Tree A)
    hdl (branch cid l r)  = branch cid <$> hdl l <*> hdl r
    hdl (leaf v)          = v -- no recursive call due to fold
    hdl failed            = var failed
\end{code}
\end{AgdaAlign}
In terms of Haskell type classes the above implementation generalizes if $C$ is
a traversable monad, because \AgdaFunction{hdl} corresponds to
\mintinline{haskell}{fmap join . sequence}.
This is quite a strong condition, but it does not seem necessary, because we
will see another example in Section \ref{scoped-algebra:state}, which does not
follow this pattern.
Furthermore, the carrier types of effects are often simple data structures like
products, coproducts, lists or trees, which are usually traversable and often
monads.
It is also unclear if a monad or just a notion of flattening is needed.


Notice that \AgdaFunction{hdl} is similar to the handler function from
Chapter \ref{chapter:higher-order}.
Because the \AgdaFunction{fold} already interpreted parts of the program we do
not call the handler recursively, but just join the results.

\begin{code}
runNondet : Prog (Nondet ∷ effs) A → Prog effs (List A)
runNondet p = dfs empty <$> runNondet′ p

fail : ⦃ Nondet ∈ effs ⦄ → Prog effs A
fail = Op (failˢ , λ())

_⁇_ : ⦃ Nondet ∈ effs ⦄ → Prog effs A → Prog effs A → Prog effs A
p ⁇ q = Op (⁇ˢ nothing , (if_then p else q))
\end{code}

\section{Exceptions}

As our second example for a scoped effect in the modular setting we take a look
at exceptions.
The syntax for \AgdaFunction{throw} is the same as before.
Similar to Chapter \ref{chapter:higher-order} \AgdaFunction{catch} has two
sub-computations, the program in scope and the handler.
The boilerplate code for the syntax is given below.

\begin{code}
data Throwˢ (E : Set)  : Set where throwˢ : (e : E) → Throwˢ E
data Catchˢ            : Set where catchˢ : Catchˢ
data Catchᵖ (E : Set)  : Set where
  mainᵖ    : Catchᵖ E
  handleᵖ  : (e : E) → Catchᵖ E

Exc : Set → Effect
Ops   (Exc E) = Throwˢ E  ↝ ⊥
Scps  (Exc E) = Catchˢ    ↝ Catchᵖ E

pattern Throw  e = (inj₁ (throwˢ e) , _)
pattern Catch  κ = (inj₁ catchˢ , κ)
\end{code}
The first part of the handler is the same as in the higher order setting.
To interpret the scoped operation \AgdaFunction{catch} we first execute scoped
program in the \AgdaInductiveConstructor{mainᵖ} position.
It produces either an exception or the result for the rest of the program.
In the later case we just return the result.
Notice that the recursive call was taken care of by the \AgdaFunction{fold}.
In the other case we obtain an exception \AgdaArgument{e} with which we can
obtain the result of the continuation in the \AgdaInductiveConstructor{handleᵖ}
position i.e. handle the exception.
The result of the exception handler is again wrapped in \AgdaDatatype{\_⊎\_}.
We pass the result along by either unwrapping the program or re-injecting the
exception in the program using \AgdaFunction{pure}.
The same function is used to traverse foreign scopes.
Again the function corresponds to the handler from Chapter
\ref{chapter:higher-order}, but without the recursive call.

\begin{code}
runExc : Prog (Exc E ∷ effs) A → Prog effs (E ⊎ A)
runExc {E} {effs} {A} = fold (λ i → (Prog effs ∘ (E ⊎_)) ^ i $ A) 1 id
  (pure ∘ inj₂)
  ( λ where
    (Throw e)    → pure (inj₁ e)
    (Other s κ)  → op (s , κ)
  ) λ where
    (Catch κ) → κ mainᵖ >>= λ where
      (inj₁ e)  → κ (handleᵖ e) >>= [ pure ∘ inj₁ , id ]
      (inj₂ x)  → x
    (Other s κ) → scp (s , λ p → [ pure ∘ inj₁ , id ] <$> κ p)
\end{code}
The smart constructors for the operations follow the known pattern.

\begin{code}
throw : ⦃ Exc E ∈ effs ⦄ → E → Prog effs A
throw e = Op (throwˢ e , λ())

_catch_ : ⦃ Exc E ∈ effs ⦄ → Prog effs A → (E → Prog effs A) → Prog effs A
p catch h = Scp $ catchˢ , λ where
  mainᵖ        → pure <$> p
  (handleᵖ e)  → pure <$> h e
\end{code}
\paragraph{Examples}
Using the smart constructors, we can define programs using exceptions as well as
nondeterminism.
Because \AgdaDatatype{Nondet} has no scoping operation, only one of the two
cases for traversing foreign scopes is used, the one for \AgdaDatatype{Nondet}
in the example with global exceptions.
More examples for interacting of different effects with scoping operations can
be found in the repository.

\begin{code}
interaction : ⦃ Nondet ∈ effs ⦄ → ⦃ Exc ⊤ ∈ effs ⦄ → Prog effs ℕ
interaction = (throw tt ⁇ pure 2) catch λ tt → pure 1
\end{code}
Running the handler for exceptions first results in local exceptions, i.e. each
nondeterministic computation branch returns either an exception or a value and
the \AgdaFunction{catch} interacts with each branch in isolation.
Therefore, the \AgdaFunction{catch} provides the alternative result
\AgdaNumber{1} on the first branch and does nothing on the second branch.

\begin{center}
\begin{code}[inline,hide]
localExc :
\end{code}
\begin{code}[inline]
 run (runNondet (runExc interaction)) ≡ inj₂ 1 ∷ inj₂ 2 ∷ []
\end{code}
\begin{code}[inline,hide]
localExc = refl
\end{code}
\end{center}
Because the program in scope of the \AgdaFunction{catch} encounters an
exception, all results are discarded and replaced with the result of the handler.
The handler for \AgdaFunction{Nondet} is run before the handler for exceptions
and therefore has to traverse the scope of \AgdaFunction{catch}.

\begin{center}
\begin{code}[inline,hide]
globalExc :
\end{code}
\begin{code}[inline]
 run (runExc (runNondet interaction)) ≡ inj₂ (1 ∷ [])
\end{code}
\begin{code}[inline,hide]
globalExc = refl
\end{code}
\end{center}

\begin{code}
\end{code}

\section{State}
\label{scoped-algebra:state}

To implement the sharing effect as described by Bunkenburg we also have to
implement the \AgdaDatatype{State} effect.
The syntax is the same as before.

\begin{code}
data Stateˢ (S : Set) : Set where putˢ : (s : S) → Stateˢ S ; getˢ : Stateˢ S

State : Set → Effect
Ops   (State S) = Stateˢ S ▷ λ{ (putˢ s) → ⊤ ; getˢ → S }
Scps  (State S) = Void

pattern Get κ     = (inj₁ getˢ , κ)
pattern Put s₁ κ  = (inj₁ (putˢ s₁) , κ)
\end{code}
In Chapter \ref{chapter:first-order} the state handler was defined recursively
and the current state was passed as an additional parameter to the handler.
When implementing the handler as a fold this is not an option.
Instead, we choose the usual carrier for state (transformer) as carrier for our handler i.e.
function from the current state to a (lifted) pair, consisting of the final
state and the result.
Handler with a function as carrier type are called \textit{parameter passing
  handler}~\cite{DBLP:conf/esop/PlotkinP09}.
When implementing the handler as a fold, the function argument appears as an
additional parameter alongside the currently handled operation and as
additional parameter to the continuation.
Parameter passing handlers are a common pattern when implementing handler,
because they allow threading a state through the handler.

When traversing a foreign scope, we have to eliminate the intermediate context.
\AgdaArgument{κ} produces a value of type
\AgdaDatatype{Prog}\AgdaSpace{}\AgdaArgument{effs}\AgdaSpace{}$($%
\AgdaArgument{S}\AgdaSpace{}\AgdaSymbol{×}\AgdaSpace{}$($%
\AgdaArgument{S}\AgdaSpace{}\AgdaSymbol{→}\AgdaSpace{}%
\AgdaDatatype{Prog}\AgdaSpace{}\AgdaArgument{effs}\AgdaSpace{}\AgdaSymbol{⋯}$)$$)$.
The lifted pair consists of the final state of the program in scope and the
function mapping an initial state to the result of the corresponding result of
the continuation.
To remove the inner context we simply pass the state along, by applying the
function to the given state using \AgdaFunction{eval}.

\begin{code}
runState : Prog (State S ∷ effs) A → S → Prog effs (S × A)
runState {S} {effs} {A} = fold (λ i → (λ X → S → Prog effs (S × X)) ^ i $ A) 1 id
  (λ x s₀ → pure (s₀ , x))
  (λ where
    (Put s₁ κ)   _   → κ tt s₁
    (Get κ)      s₀  → κ s₀ s₀
    (Other s κ)  s₀  → op (s , λ p → κ p s₀)
  ) λ where
    (Other s κ)  s₀  → scp (s , λ p → eval <$> κ p s₀)
  where
    eval : ∀ {A B} → A × (A → B) → B
    eval (a , f) = f a

evalState : Prog (State S ∷ effs) A → S → Prog effs A
evalState s₀ p = π₂ <$> runState s₀ p
\end{code}
The operations are just the usual generic operations for the effect.

\begin{code}
get : ⦃ State S ∈ effs ⦄ → Prog effs S
get = Op (getˢ , pure)

put : ⦃ State S ∈ effs ⦄ → S → Prog effs ⊤
put s = Op (putˢ s , pure)
\end{code}


\section{Share}

Lastly we implement the sharing effect.
The signature contains just the single unary, scoped operation, \AgdaFunction{share}
which creates new sharing scope.

\begin{code}
data Shareˢ : Set where shareˢ : SID → Shareˢ

Share : Effect
Ops   Share = Void
Scps  Share = Shareˢ ↝ ⊤

pattern ShareScp sid κ = (inj₁ (shareˢ sid) , κ)
\end{code}
The handler is implemented as a parameter passing handler.
The carrier type is a function, which maps from the current choice id to the
shared program.
In contrast to \AgdaDatatype{State}, where a function was the intuitive solution,
\AgdaFunction{runShare′} follows the explanation by
\textcite{DBLP:conf/esop/PlotkinP09} more closely.
It simulates handler for \AgdaDatatype{Share} which passes additional parameters
around.
\AgdaFunction{runShare} supplies initial parameters, hiding the parameters from
the caller.

When handling a sharing scope, the captured program is handled with the id stored
in the shape of the scoping operation.
The continuation is accessed via \AgdaFunction{>>=} and continuous with the
outside id.
Scopes of foreign effects do not affect sharing.
The foreign scope is a subscope of the current sharing scope.
Therefore, the choices are labelled with the same scope id.
To guaranty the uniqueness of ids we have to thread the current choice id through
the inner scope.
We add the current choice id as an additional return value.
After traversing a foreign scope, we continue with the last id from the inner
scope.
In case of a sharing scope we number choices inside the scope using the scopes
id and continue with the current outer id after the scope.

\begin{code}
runShare′ : ⦃ Nondet ∈ effs ⦄ → Prog (Share ∷ effs) A → 
  Maybe CID → Prog effs (Maybe CID × A)
runShare′ {effs} {A} ⦃ p ⦄ = fold 
  (λ i → ((λ X → Maybe CID → Prog effs (Maybe CID × X)) ^ i) A) 1 id
  (λ z cid → var (cid , z))
  (λ{ (Other s pf) → case prj (opsInj p) (s , pf) of λ where
    nothing              sid                   → op (s , λ p → pf p sid)
    (just (failˢ  , κ))  sid                   → fail
    (just (⁇ˢ _   , κ))  nothing               → Op (⁇ˢ nothing , λ p → κ p nothing)
    (just (⁇ˢ _   , κ))  cid@(just (sid , n))  → Op (⁇ˢ cid , λ p → κ p (just (sid , suc n)))
  }) λ where
    (ShareScp sid′ κ) sid → κ tt (just $ sid′ , 0) >>= λ (_ , r) → r sid
    (Other    s    κ) sid → scp (s , λ p → (λ (sid′ , k) → k sid′) <$> κ p sid)

runShare : ⦃ Nondet ∈ effs ⦄ → Prog (Share ∷ effs) A → Prog effs A
runShare p = π₂ <$> runShare′ p nothing
\end{code}
We introduce the helper function \AgdaFunction{runShare} to conveniently call
the actual sharing handler.
The handler is initially called without an id, because by default choices are
not shared.
Furthermore, after evaluating all sharing syntax we discard the current id,
because it is not needed anymore.

\begin{code}[hide]
record Shareable (effs : List Effect) (A : Set) : Set where
  field shareArgs : A → Prog effs A
open Shareable ⦃...⦄ public

instance
  ℕ-Shareable : Shareable effs ℕ
  Shareable.shareArgs ℕ-Shareable = var

  Bool-Shareable : Shareable effs Bool
  Shareable.shareArgs Bool-Shareable = var

record Normalform (effs : List Effect) (A B : Set) : Set where
  field nf : A → Prog effs B

  infix 10 !_
  !_ : Prog effs A → Prog effs B
  !_ = _>>= nf
open Normalform ⦃...⦄ public

instance
  ℕ-Normalform : Normalform effs ℕ ℕ
  Normalform.nf ℕ-Normalform = var

  Bool-Normalform : Normalform effs Bool Bool
  Normalform.nf Bool-Normalform = var

  ⊤-Normalform : Normalform effs ⊤ ⊤
  Normalform.nf ⊤-Normalform = var
\end{code}

The \AgdaDatatype{Shareable} and \AgdaDatatype{Normalform} infrastructure from
Chapter \ref{chapter:first-order} can be reused to define the
\AgdaFunction{share} operator.
The program in the scope is again captured by mapping \AgdaFunction{pure} over it.
The construction of scope ids follows again the implementation by
\textcite{bunkenburg2019modeling}.

\begin{code}
share⟨_⟩ : ⦃ Share ∈ effs ⦄ → SID → Prog effs A → Prog effs A
share⟨_⟩ sid p = Scp (shareˢ sid , λ _ → pure <$> p)

share : ⦃ State SID ∈ effs ⦄ → ⦃ Share ∈ effs ⦄ → ⦃ Shareable effs A ⦄ →
  Prog effs A → Prog effs (Prog effs A)
share p = do
    (i , j) ← get
    put (i + 1 , j)
    let p′ = do
          put (i , j + 1)
          x   ← p
          x′  ← shareArgs x
          put (i + 1 , j)
          pure x′
    pure $ share⟨ i , j ⟩ p′
\end{code}

\paragraph{Examples}
Using \AgdaFunction{Nondet}, \AgdaFunction{State} and \AgdaFunction{Share} we
can again simulate call-time choice semantics.
We introduce the function \AgdaFunction{runCTC} to easily evaluate programs
using normalizable data and call-time choice semantics.

\begin{code}
runCTC : ⦃ Normalform (State SID ∷ Share ∷ Nondet ∷ []) A B ⦄ →
  Prog (State SID ∷ Share ∷ Nondet ∷ []) A → List B
runCTC p = run $ runNondet $ runShare $ evalState (! p) (1 , 1)

coin : ⦃ Nondet ∈ effs ⦄ → Prog effs ℕ
coin = pure 0 ⁇ pure 1

doubleCoin : ⦃ Nondet ∈ effs ⦄ → ⦃ Share ∈ effs ⦄ → ⦃ State SID ∈ effs ⦄ →
  Prog effs ℕ
doubleCoin = do c ← share coin
                ⦇ c + c ⦈
\end{code}
As expected, doubling a shared coin yields the results $0$ and $2$.
\begin{center}
\begin{code}[inline,hide]
runDoubleCoin :
\end{code}
\begin{code}[inline]
 runCTC doubleCoin ≡ 0 ∷ 2 ∷ []
\end{code}
\begin{code}[inline,hide]
runDoubleCoin = refl
\end{code}
\end{center}
In contrast to Chapter \ref{chapter:higher-order} this approach can also model
deep effects without involving universe levels.
The definition of an effectful list as well as its type class instances from
Chapter \ref{chapter:first-order} are reused.
The definition of \AgdaFunction{doubleHead} stays the same, while the new
underlying representation for the effects eliminates the problem of potentially
mismatched scope delimiters.

\begin{code}[hide]
data Listᴹ (effs : List Effect) (A : Set) : {Size} → Set where
  nilᴹ  : ∀ {i} → Listᴹ effs A {i}
  consᴹ : ∀ {i} → Prog effs A → Prog effs (Listᴹ effs A {i}) → Listᴹ effs A {↑ i}

pattern []ᴹ         = var nilᴹ
pattern _∷ᴹ_ mx mxs = var (consᴹ mx mxs)

headᴹ : ⦃ Nondet ∈ effs ⦄ → Prog effs (Listᴹ effs A) → Prog effs A
headᴹ = _>>= λ where
  nilᴹ         → fail
  (consᴹ mx _) → mx

instance
  Listᴹ-Normalform : ∀ {i} → ⦃ Normalform effs A B ⦄ → Normalform effs (Listᴹ effs A {i}) (List B)
  Normalform.nf Listᴹ-Normalform nilᴹ           = var []
  Normalform.nf Listᴹ-Normalform (consᴹ mx mxs) = ⦇ ! mx ∷ ! mxs ⦈

  Listᴹ-Shareable : ∀ {i} → ⦃ Shareable effs A ⦄ → ⦃ State SID ∈ effs ⦄ → ⦃ Share ∈ effs ⦄ → Shareable effs (Listᴹ effs A {i})
  Shareable.shareArgs Listᴹ-Shareable nilᴹ           = []ᴹ
  Shareable.shareArgs Listᴹ-Shareable (consᴹ mx mxs) = ⦇ consᴹ (share mx) (share mxs) ⦈
\end{code}
\begin{code}
doubleHead : ⦃ Nondet ∈ effs ⦄ → ⦃ Share ∈ effs ⦄ → ⦃ State SID ∈ effs ⦄ →
  Prog effs ℕ
doubleHead = do mxs ← share (coin ∷ᴹ []ᴹ)
                ⦇ headᴹ mxs + headᴹ mxs ⦈
\end{code}
Using the \AgdaDatatype{Shareable} instance we can share choices inside data
structures.
As expected, sharing a coin inside a list, unwrapping and doubling it yields the
results \AgdaNumber{0} and \AgdaNumber{2}.

\begin{center}
\begin{code}[inline,hide]
runDoubleHead :
\end{code}
\begin{code}[inline]
 runCTC doubleHead ≡ 0 ∷ 2 ∷ []
\end{code}
\begin{code}[inline,hide]
runDoubleHead = refl
\end{code}
\end{center}


\section{Results}

In this chapter we presented an implementation of scoped
algebras~\cite{DBLP:conf/lics/PirogSWJ18} in Agda.
To generally guaranty termination of recursive functions on values of
\AgdaDatatype{Prog}, we defined its recursion scheme based on the work by
\textcite{DBLP:journals/corr/abs-1806-05230}.
Our recursion scheme, based on \textcite{DBLP:journals/corr/abs-1806-05230}'s idea
of introducing an index type, has essentially the same shape as the one needed
for scoped algebras.

Using the recursion scheme we were able to define the nondeterminism example by
\textcite{DBLP:conf/lics/PirogSWJ18} in Agda, without resorting to sized types.
Furthermore, using the corresponding induction scheme for \AgdaDatatype{Prog} we
were able to prove proposition by induction on \AgdaDatatype{Prog}.
With this approach we could easily implement deep effects, because the
\AgdaDatatype{Prog} monad does not increase the universe level.
Based on the patterns from the earlier chapters i.e. the general idea of
partially removing syntax~\cite{DBLP:conf/haskell/SchrijversPWJ19}, we
implemented a naive modularization for handlers.
In our current limited, practical setting the examples in the thesis as well as
the ones in the repository seem promising.
The cases for traversing foreign sharing scopes still bear resemblance to the
functions passed to \AgdaFunction{handle} in Chapter \ref{chapter:higher-order}.
This is partially expected, because the monad in this chapter is derived from a
monad similar to the one in Chapter \ref{chapter:higher-order}.
However, due to rewriting the monad, we again loose explicit control over the
continuation.
Using our modular approach we could again implement
Bunkenburg's~\cite{bunkenburg2019modeling} sharing handler.
