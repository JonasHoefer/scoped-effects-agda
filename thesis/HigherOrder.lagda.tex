\begin{code}[hide]
{-# OPTIONS --allow-unsolved-metas #-}

module HigherOrder where

open import Function using (_∘_; id; case_of_)

open import Data.Product using (Σ-syntax; _,_)
open import Data.Maybe using (Maybe; fromMaybe; nothing; just)

open import Relation.Unary using (_⊆_)
open import Relation.Binary.PropositionalEquality using (_≡_; refl; cong; sym; trans)

variable
  A B C : Set
\end{code}

% \begin{itemize}
%   \item ultimatly failed
%   \item based on second approach from ``Effect Handlers in
%   Scope''\cite{DBLP:conf/haskell/WuSH14}
%   \item partially working implementation using (simplified) indexed containers
%   \item scoped effects without effectful data structures are working
%   \item does only partially work with effectful data structures due to size
%     problems when using scoped effects like sharing
%   \item sharing is the ``worst case scenario'' (scoped effect executed on every
%     effectful component of data structure)
%   \item size problems generally occur with this approach thanks to existential
%     types (instantiated with again type \texttt{Free ...}) $\Rightarrow$
%     scoped effect representation (indexed container; indexed bicontainer;
%     in \texttt{Free} inlined existential type variable; ...) does not matter
%   \item see \url{https://lists.chalmers.se/pipermail/agda/2020/012248.html} for
%     simple implementation and explanation
%   \item embedding a growing hierarchy of types into each effectful programs
%     seems impractical (explanation in email)
% \end{itemize}

To address problems of the first order approach, like mismatched scope
delimiters and explicit control over the continuation, Wu et al. introduced a
second approach utilizing higher order syntax~\cite{DBLP:conf/haskell/WuSH14}.
Bunkenburg already tried to implement some scoped effects using this approach,
but failed due to limitations of Coq~\cite{bunkenburg2019modeling}.

Chapter \ref{chapter:first-order} demonstrated how the first order approach
could be transferred to Agda utilizing container representations for functors
and sized types.
This chapter focuses on the implementation of effects using higher order syntax.
It partially follows Bunkenburg's approach, because the limitation of Coq
described by him does not exist in Agda.
Therefore this chapter focuses on expanding his approach further.
Due to other limitations, needed for the consistency of Agda, this approach
ultimately failed again.


\section{Higher Order Syntax}

In the first order approach effects only store possible continuations.
In the higher order approach by Wu et al. scoped effects store possible
continuations as well as the part of the program in their scope.

In the first order approach the functors, which describe the syntax, are applied
to types of the form \texttt{Free F A} i.e. programs consisting of syntax as
described by \texttt{Free F}, which produce values of type \texttt{A}.
In the higher order approach these functors are generalized to higher order
functors i.e. in Haskell a data type of kind \texttt{(* -> *) -> (* -> *)}.
These functors are applied to \texttt{Free F} and \texttt{A} separately.

Because the program type is separated in syntax and result type it is possible
to store programs, which use the same signature but produce values of different
types.
Consider the following definition of the higher order exception syntax as given
by Wu et al.

\begin{minted}{haskell}
    data HExc e m a = Throw e | forall x. Catch (m x) (e -> m x) (x -> m a)
\end{minted}
The catch operation stores the computation in the scope of the catch block
\texttt{m x}, the handler producing an alternative value in case of an error
\texttt{e -> m x} as well as the continuation \texttt{x -> m a}.
The three programs agree on an arbitrary, but per \texttt{catch} fixed type
\texttt{x}.

Calling \AgdaFunction{>>=} on a higher order term just extends the continuation
i.e. just substitutes the variables in the trees of type \texttt{m a}.
See figure \ref{higher-order:syntax:bind} for an example.
Therefore these operations aren't algebraic but also capture the notation of
scopes.

\begin{figure}
  \begin{tikzpicture}
    \node[] {$\bind$};
  \end{tikzpicture}
  \caption{calling \bind{} on operations with and without scopes (TODO)}
  \label{higher-order:syntax:bind}
\end{figure}


\section{Representing Strictly Positive Higher Order Functors}

In Haskell the generalization to higher order syntax is straight forward.
In Agda we again have to find an appropriate representation for functor to
guaranty strict positivity, which leads to the idea of an indexed container as
described by Altenkirch et al.~\cite{DBLP:journals/jfp/AltenkirchGHMM15}
\footnote{Note that this is not the first publication related to indexed
  containers. Papers of the same name, by the same author(s) exist, as well as
  data types, known as ``Interaction Structures'' with essentially the same shape.}.

Indexed containers from the paper as well in the Agda standard library are
defined as follows\footnote{The names of the fields are changed to highlight
  differences between the different definitions.}.

\begin{code}
record IndexedContainer (I O : Set₁) : Set₁ where
  field
    Shape : O → Set
    Pos : ∀ {o} → Shape o → Set
    Ctx : ∀ {o} → (s : Shape o) → Pos s → I

  ⟦_⟧ : (I → Set) → (O → Set)
  ⟦_⟧ F A = Σ[ s ∈ Shape A ] ((p : Pos s) → F (Ctx s p))
\end{code}
Allowing indicies of arbitrary levels and instantiating \AgdaArgument{I} and
\AgdaArgument{O} with \AgdaDatatype{Set} gives rise to a container extension
of the correct shape.

% \begin{AgdaAlign}
The above definition is similar to Bunkenburg's first definition of a container
representation for a higher order functor.
His definition is given below.
\begin{code}
record HContainer : Set₂ where
  field
    Shape : Set₁
    Pos : Shape → Set
    Ctx : (s : Shape) → Pos s → Set → Set

  ⟦_⟧ : (Set → Set₁) → (Set → Set₁)
  ⟦_⟧ F A = Σ[ s ∈ Shape ] ((p : Pos s) → F (Ctx s p A))
\end{code}
In the second definition the \AgdaField{Shape} and \AgdaField{Pos} types cannot
depend on the given index i.e. the argument for the returned functor.
Otherwise, the both definitions are quite similar.
All stored elements are lifted by the given functor \AgdaArgument{F}.
The type of the \AgdaArgument{F}'s argument is generated using \AgdaField{Ctx}.
\AgdaField{Ctx} has access to the chosen shape, accessed position and the 
argument type \AgdaArgument{A}.
Conceptually, \AgdaField{Ctx} is used to choose between the stored
\AgdaArgument{X} and the argument type \AgdaArgument{A} i.e. programs inside or
outside the operations scope.

% TODO: explain ⊆
We can define \AgdaFunction{hmap} for both, which preserves identity and
composition.
As noted by Bunkenburg, we cannot define \AgdaFunction{emap} for all extensions
of a such a higher-order functor i.e. the produced type constructor is not
necessarily a functor.

\begin{code}
  hmap : {F G : Set → Set₁} → (F ⊆ G) → (⟦_⟧ F ⊆ ⟦_⟧ G)
  hmap α (s , pf) = s , α ∘ pf

  emap : {F : Set → Set₁} → (F A → F B) → ⟦_⟧ F A → ⟦_⟧ F B
  emap f (s , pf) = s , λ p → {! it's not clear if the type of pf p involves A!}
\end{code}
% \end{AgdaAlign}
Similar to the Haskell implementation by Wu et al. we could define a type class
to implement \AgdaFunction{emap} for all valid higher-order functors.
Both representations share another problem, the produced type constructor is not
necessarily strictly positive.
This is a necessary requirement to define inductive, effectful data structures.

Bunkenburg fixed this problem by switching to version of a Bi-Container.
A Bi-Container has two position types for each shape.
Its extension is a Bifunctor, which is strictly positive in both arguments.
One set of positions can be used to store computations producing values of type
\AgdaArgument{A}, while the (using the indexing mechanism from above) stores
computations producing values involving the captured type \AgdaArgument{X}.
Defining inductive, effectful data types with this approach isn't a problem,
because the positions for \AgdaArgument{A} are used identical to the first-order
approach.

Alternatively, one could fix the positivity problem by returning a first order
container instead of a type constructor from \AgdaField{Ctx}.
In most effects \AgdaArgument{A} is only used directly i.e. the Haskell functor
contains \texttt{a} only as \texttt{m a}.
For the following simple examples we can therefore simplify \AgdaField{Ctx}
further to avoid complicated representations, which aren't needed.
Alternatively, we could use the indexed Bi-Container.
Going forward we will use the simplified indexed container, because it resembles
the Haskell implementation more closely.

\begin{code}
record Container : Set₂ where
  constructor _◁_/_
  field
    Shape : Set₁
    Pos : Shape → Set
    Ctx : (s : Shape) → Pos s → Maybe Set₁

  ⟦_⟧ : (Set₁ → Set₁) → Set₁ → Set₁
  ⟦_⟧ F A = Σ[ s ∈ Shape ] ((p : Pos s) → F (fromMaybe A (Ctx s p)))

  hmap : {F G : Set₁ → Set₁} → (F ⊆ G) → (⟦_⟧ F ⊆ ⟦_⟧ G)
  hmap α (s , pf) = s , α ∘ pf

  emap : {F : Set₁ → Set₁} {X Y : Set₁} → (F X → F Y) → ⟦_⟧ F X → ⟦_⟧ F Y
  emap {F} {X} {Y} f (s , pf) = s , λ p → helper s p (pf p)
    where helper : (s : Shape) (p : Pos s) → 
            F (fromMaybe X (Ctx s p)) → F (fromMaybe Y (Ctx s p))
          helper s p x with Ctx s p
          ... | nothing = f x
          ... | just _  = x
open Container
\end{code}
Ultimately the choice of representation for the functor doesn't matter, because
all approaches based on storing a type directly encounter the problem described
in section \ref{higher-order:eff-data-ext-types}.

Using our choice of functor representation we can define the free monad for
higher order syntax.
As in Haskell, the argument for the functor in the
\AgdaInductiveConstructor{impure} case is split.

\begin{code}
data Prog (C : Container) (A : Set₁) : Set₁ where
  pure    : A → Prog C A
  impure  : ⟦ C ⟧ (Prog C) A → Prog C A
\end{code}
Notice that due to the potentially captured type in the \AgdaField{Shape},
stored in the container extension, the universe level for \AgdaDatatype{Prog}
increased to $1$.
This allows us to increase the universe level of \AgdaArgument{A} to one,
allowing us to define effectful data structures.


\section{Effectful Data and Existential Types}
\label{higher-order:eff-data-ext-types}

A central problem with this approach is that it cannot model deep effects
without involving variable levels.
Consider the the following definition of an effectful list.

\begin{code}
data Listᴹ (C : Container) (A : Set₁) : Set₁ where
  nil   : Listᴹ C A
  cons  : Prog C A → Prog C (Listᴹ C A) → Listᴹ C A
\end{code}
\AgdaDatatype{Listᴹ}\AgdaSpace{}\AgdaArgument{C}\AgdaSpace{}\AgdaArgument{A} is
an element of \AgdaDatatype{Set₁}, because it stores elements of type
\AgdaDatatype{Prog}\AgdaSpace{}\AgdaArgument{C}\AgdaSpace{}\AgdaArgument{X},
which is an element of \AgdaDatatype{Set₁}.
\AgdaDatatype{Prog}\AgdaSpace{}\AgdaArgument{C}\AgdaSpace{}\AgdaArgument{X} is
an element of \AgdaDatatype{Set₁}, because it holds values of the container
extension i.e. elements of the \AgdaField{Shape} type, which is an element
\AgdaDatatype{Set₁}, because it has to store the result type of the
subcomputation \AgdaArgument{X}.
Therefore the
\AgdaDatatype{Listᴹ}\AgdaSpace{}\AgdaArgument{C}\AgdaSpace{}\AgdaArgument{A} is
to large to be stored as result type of a subcomputation.
Therefore it is not possible to call a scoped operation (such as
\AgdaFunction{share}) on an effectful data structure.

More generally speaking, let us fix the universe level of the result type $X$ of
the subcomputation $\ell_X = l$.
The type is stored in the \AgdaField{Shape}, therefore the universe level of the
\AgdaField{Shape} type is given by $\ell_S = l + 1$.
Because the \AgdaDatatype{Prog} stores an element of the \AgdaField{Shape} its
level is given by $\ell_P = \ell_S$.
Effectful data structures live in the same universe as \AgdaDatatype{Prog}, but
are simultaneously required to be stored in the subcomputation.
We obtain the contradictory requirements $\ell_E = \ell_P = \ell_S = l + 1$ and
$\ell_E  \leqslant l$.

Holding on to the idea of storing types in the \AgdaField{Shape}, the only way to
fix this problem is to let all levels vary as needed.
This leads to different levels across the program, which are different to unify.
Furthermore, when mapping a scoped operation over a recursive data structure
(e.g. \AgdaFunction{share}) the increase in the universe level depends on the
depth of the structure.
Both of these phenomena are hard to deal with and in a complex program
potentially unsolvable.


\section{The Problem with Indexing}

One could try solving the problem of existential types by indexing the program
over a typing context.
The level of the index is independent from the level of the structure, therefore
the chain of reasoning from section \ref{higher-order:eff-data-ext-types} does
not apply.
The typing context needs to have the same structure as the program.

A central problem with this approach is that every call of \AgdaFunction{>>=}
changes the structure of the program and therefore the context.
These changes now have to be reflected on the type level, because the typing
context has to reflect the programs shape.
If \AgdaFunction{>>=} substitutes leaves differently, the changes to the context
aren't simple substitutions, but depend on the stored value.
As with increasing universe levels, this complications are increasingly hard to
deal with and are hard to hide from the user of the library.

