\documentclass[10pt,a4paper,twoside,notitlepage]{report}

\usepackage[english,ngerman]{babel}
\usepackage[a4paper,left=30mm,right=30mm,width=150mm,top=25mm,bottom=25mm]{geometry}
% \usepackage[twoside, inner=28mm, outer=37mm, top=35mm, bottom=49mm]{geometry}

\usepackage{fontspec}
\usepackage{unicode-math}
\usepackage{wasysym}
\setmathfont{Latin Modern Math}                 % normal math font
\setmathfont[range={\lBrace,\rBrace,\llparenthesis,\rrparenthesis}]{XITSMath}
% fallback for these 4 glyphs
\setsansfont{Latin Modern Sans}
% \setmainfont{Latin Modern Roman}
% \setmonofont{Fira Code}

\usepackage{newunicodechar}
\newunicodechar{ℕ}{\ensuremath{\mathnormal{\mathbb{N}}}}
\newunicodechar{∷}{\ensuremath{\mathnormal{::}}}
\newunicodechar{⊔}{\ensuremath{\mathnormal{\sqcup}}}
\newunicodechar{∀}{\ensuremath{\mathnormal{\forall}}}
\newunicodechar{ℓ}{\ensuremath{\mathnormal{\ell}}}
\newunicodechar{𝒰}{\ensuremath{\mathnormal{\mathcal{U}}}}
\newunicodechar{▷}{\ensuremath{\mathnormal{\rhd}}}
\newunicodechar{◁}{\ensuremath{\mathnormal{\lhd}}}
\newunicodechar{∈}{\ensuremath{\mathnormal{\in}}}
\newunicodechar{⦃}{\ensuremath{\mathnormal{\lBrace}}}
\newunicodechar{⦄}{\ensuremath{\mathnormal{\rBrace}}}
\newunicodechar{⦇}{\ensuremath{\mathnormal{\llparenthesis}}}
\newunicodechar{⦈}{\ensuremath{\mathnormal{\rrparenthesis}}}
\newunicodechar{⟨}{\ensuremath{\mathnormal{\langle}}}
\newunicodechar{⟩}{\ensuremath{\mathnormal{\rangle}}}
\newunicodechar{≅}{\ensuremath{\mathnormal{\cong}}}
\newunicodechar{⊕}{\ensuremath{\mathnormal{\oplus}}}
\newunicodechar{⊎}{\ensuremath{\mathnormal{\uplus}}}
\newunicodechar{⊆}{\ensuremath{\mathnormal{\subseteq}}}
\newunicodechar{α}{\ensuremath{\mathnormal{\alpha}}}
\newunicodechar{η}{\ensuremath{\mathnormal{\eta}}}
\newunicodechar{σ}{\ensuremath{\mathnormal{\sigma}}}
\newunicodechar{κ}{\ensuremath{\mathnormal{\kappa}}}
\newunicodechar{λ}{\ensuremath{\mathnormal{\lambda}}}
\newunicodechar{π}{\ensuremath{\mathnormal{\pi}}}
\newunicodechar{ω}{\ensuremath{\mathnormal{\omega}}}
\newunicodechar{⊛}{\ensuremath{\mathnormal{\circledast}}}
\newunicodechar{∘}{\ensuremath{\mathnormal{\circ}}}
\newunicodechar{≡}{\ensuremath{\mathnormal{\equiv}}}
\newunicodechar{⁇}{\ensuremath{\mathnormal{?\!?}}}
\newunicodechar{∎}{\ensuremath{\mathnormal{\QED}}}
\newunicodechar{⋯}{\ensuremath{\mathnormal{\dots}}}
\newunicodechar{↝}{\ensuremath{\mathnormal{\leadsto{}}}}
% see https://tex.stackexchange.com/questions/511401/newunicodechar-fails-for-prime-only
\AtBeginDocument{\newunicodechar{′}{\makebox[\fontcharwd\font`a]{$\prime$}}}
\AtBeginDocument{\newunicodechar{″}{\makebox[\fontcharwd\font`a]{$\dprime$}}}
\newunicodechar{⊥}{\ensuremath{\mathnormal{\bot}}}
\newunicodechar{⊤}{\ensuremath{\mathnormal{\top}}}
\newunicodechar{ᶜ}{\textsuperscript{c}}
\newunicodechar{ˡ}{\textsuperscript{l}}
\newunicodechar{ₙ}{\textsubscript{n}}
\newunicodechar{ᵖ}{\textsuperscript{p}}
\newunicodechar{ʳ}{\textsuperscript{r}}
\newunicodechar{ˢ}{\textsuperscript{s}}
\newunicodechar{ˣ}{\textsuperscript{x}}
\newunicodechar{ᴵ}{\textsuperscript{I}}
\newunicodechar{ᴰ}{\textsuperscript{D}}
\newunicodechar{ᴸ}{\textsuperscript{L}}
\newunicodechar{ᴹ}{\textsuperscript{M}}
\newunicodechar{ᴾ}{\textsuperscript{P}}
\newunicodechar{ᵂ}{\textsuperscript{W}}
\newunicodechar{₀}{\textsubscript{0}}
\newunicodechar{₁}{\textsubscript{1}}
\newunicodechar{₂}{\textsubscript{2}}
\newunicodechar{₃}{\textsubscript{3}}
\newunicodechar{₄}{\textsubscript{4}}
\newunicodechar{₅}{\textsubscript{5}}

\usepackage[hidelinks]{hyperref}
\usepackage[links]{agda}
\usepackage{mathtools}

% nicer monadic operators
\newcommand{\fmap}[0]{\ensuremath{<\hspace{.4pt}\mathclap{\raisebox{-0.85pt}{\scalebox{0.95}{$\$$}}}\hspace{.4pt}>}}
\newcommand{\fmapconst}[0]{\ensuremath{<\hspace{.4pt}\mathclap{\raisebox{-0.85pt}{\scalebox{0.95}{$\$$}}}}}
\newcommand{\ap}[0]{\ensuremath{<\hspace{.4pt}\mathclap{*}\hspace{.4pt}>}}
\newcommand{\bind}[0]{\ensuremath{\mathbin{>\!\!\!>\mkern-6.7mu=}}}
\newcommand{\kleisli}[0]{\ensuremath{\mathbin{>\!\!=\!\!\!>}}}

% replace certain character sequences with the nicer operators
\DeclareRobustCommand{\AgdaFormat}[2]{%
  \ifthenelse{\equal{#1}{>>=}}
  {\ensuremath{\mathbin{>\!\!\!>\mkern-6.7mu=}}}
  {\ifthenelse{\equal{#1}{\AgdaUnderscore{}>>=}}
  {\ensuremath{\AgdaUnderscore{}\mathbin{>\!\!\!>\mkern-6.7mu=}}}
  {\ifthenelse{\equal{#1}{>>=\AgdaUnderscore{}}}
  {\ensuremath{\mathbin{>\!\!\!>\mkern-6.7mu=}\AgdaUnderscore{}}}
  {\ifthenelse{\equal{#1}{\AgdaUnderscore{}>>=\AgdaUnderscore{}}}
  {\ensuremath{\AgdaUnderscore{}\mathbin{>\!\!\!>\mkern-6.7mu=}\AgdaUnderscore{}}}
  {\ifthenelse{\equal{#1}{>>}}
  {\ensuremath{\mathbin{>\!\!\!>}}}
  {\ifthenelse{\equal{#1}{\AgdaUnderscore{}>>\AgdaUnderscore{}}}
  {\AgdaUnderscore{}\ensuremath{\mathbin{>\!\!\!>}}\AgdaUnderscore{}}
  {\ifthenelse{\equal{#1}{<\$>}}
  {\fmap}
  {\ifthenelse{\equal{#1}{\AgdaUnderscore{}<\$\AgdaUnderscore{}}}
  {\AgdaUnderscore{}\fmapconst{}\;\;\AgdaUnderscore{}}
  {\ifthenelse{\equal{#1}{<\$}}
  {\fmapconst\;}
  {\ifthenelse{\equal{#1}{>=>}}
  {\kleisli}
  {\ifthenelse{\equal{#1}{++}}
  {\ensuremath{+\!+}}
  {\ifthenelse{\equal{#1}{\AgdaUnderscore{}++\AgdaUnderscore{}}}
  {\ensuremath{\AgdaUnderscore{}+\!+\AgdaUnderscore{}}}
  {\ifthenelse{\equal{#1}{\AgdaUnderscore{}++}}
  {\ensuremath{\AgdaUnderscore{}+\!+}}
  {\ifthenelse{\equal{#1}{++\AgdaUnderscore{}}}
  {\ensuremath{+\!+\AgdaUnderscore{}}}
  {\ifthenelse{\equal{#1}{\$}}
  {\ensuremath{\$}}
  {\ifthenelse{\equal{#1}{<\$>\AgdaUnderscore{}}}
  {\fmap\AgdaUnderscore{}}
  {\ifthenelse{\equal{#1}{Context.\AgdaUnderscore{}<\$>\AgdaUnderscore{}}}
  {Context.\AgdaUnderscore{}\fmap\AgdaUnderscore{}}
  {\ifthenelse{\equal{#1}{\AgdaUnderscore{}<\$>\AgdaUnderscore{}}}
  {\AgdaUnderscore{}\fmap\AgdaUnderscore{}}
  {\ifthenelse{\equal{#1}{\AgdaUnderscore{}<*>\AgdaUnderscore{}}}
  {\AgdaUnderscore{}\ap\AgdaUnderscore{}}
  {\ifthenelse{\equal{#1}{\AgdaUnderscore{}<*>}}
  {\AgdaUnderscore{}\ap}
  {\ifthenelse{\equal{#1}{<*>}}
  {\ap}
  {#2}}}}}}}}}}}}}}}}}}}}}}

\usepackage{graphicx}
\graphicspath{{images/}}

\usepackage{tikz}
\usetikzlibrary{shapes.geometric,arrows,fit,matrix,positioning}
\tikzset
{
  treenode/.style = {align=center, anchor=center}
}

\usepackage{fancyhdr}
\pagestyle{fancy}
\fancyhf{}%
\fancyhead[LE,RO]{\thepage}
\fancyhead[LO]{\leftmark}
\fancyhead[RE]{\rightmark}
\renewcommand{\headrulewidth}{0.4pt}
\renewcommand{\footrulewidth}{0.0pt}

% Chapter pages
\fancypagestyle{plain}{%
  \fancyhf{}%
  \fancyhead[RO,LE]{\thepage}
  \renewcommand{\headrulewidth}{0.0pt}%
  \renewcommand{\footrulewidth}{0.0pt}%
}

\usepackage{minted}

\usepackage[maxcitenames=2,style=numeric,backend=biber]{biblatex}
\addbibresource{thesis.bib}
\setcounter{biburllcpenalty}{7000}
\setcounter{biburlucpenalty}{8000}

\title{Implementing a Library for Scoped Algebraic Effects in Agda}
\author{Jonas Höfer}
\date{2020}

\begin{document}
\begin{titlepage}
  \begin{large}
    \begin{center}

      \includegraphics[width=10cm]{htwk_logo2.png}

      \vskip 2cm

      \textbf{HTWK Leipzig}\\
      Fakultät Informatik und Medien\\

      \vskip 2cm

      \textbf{Bachelor Thesis}

      \vskip 2cm

      \textbf{Implementing a Library for Scoped Algebraic Effects in Agda}\\

      \vfill

      Autor: Jonas Höfer\\

      \vskip 2cm

      \begin{tabular}{rll}
         Betreuer : & Prof. Dr. Johannes Waldmann & HTWK Leipzig\\
                    & M. Sc. Niels Bunkenburg     & Christian-Albrechts-Universität zu Kiel\\
      \end{tabular}

      \vskip 2cm

      Leipzig, \today\\

    \end{center}
  \end{large}
\end{titlepage}

\thispagestyle{plain}
\selectlanguage{english}
\begin{abstract}
  Computational effects are a central concept in functional programming as well
  as formal verification.
  Different representations for computational effects exist, notably monads and
  algebraic effects.
  The composability of the representation is important, to allow the development
  of modular programs and reasoning about them.
  Algebraic effects and handlers are generally a more modular approach than
  traditional monad transformers.
  Furthermore, algebraic effects are convenient representation of computational
  effects in total, dependently typed languages.
  For example,
  \textcite{DBLP:journals/programming/DylusCT19,DBLP:conf/haskell/ChristiansenDB19}
  used them to explicitly represent some ambient effects, such as Haskell's
  partiality, in the proof assistant Coq.

  In the traditional approach, operations with scopes can only be implemented as
  an effect handler.
  \textcite{DBLP:conf/haskell/WuSH14} noted a lack of modularity, compared to
  monad transformers, when multiple effect with scoping operations are used.
  This is an important limitation because most commonly used effects have
  associated operations with scopes.
  To fix this limitation \textcite{DBLP:conf/haskell/WuSH14} introduced scoped
  effects.
  With regards to verification, scoped effects are needed to model more
  complex ambient effects, such as sharing.

  In this thesis we explore how well three different Haskell implementations of
  scoped effects by
  \textcite{DBLP:conf/haskell/WuSH14,DBLP:conf/lics/PirogSWJ18} translate in the
  dependently typed, functional programming language Agda.
  We implement a library for scoped effects that allows the implementation of
  effectful program as well as formal reasoning using Agda's theorem proving
  capabilities.
  To test the applicability of our implementations, we implement effects related
  to modelling Curry's~\cite{Hanus95curry} ambient non-strict non-determinism.
  Our main example for a scoped effect is sharing of nondeterminism as modelled
  by \textcite{bunkenburg2019modeling} to simulating Curry's call-time choice
  semantics.
  To model ambient effects in data structures, the implementation also supports
  deep effects.
\end{abstract}
\selectlanguage{ngerman} 
\begin{abstract}
  Nebenwirkungen von Programmen sind ein zentrales Konzept in der funktionalen
  Programmierung und in der formalen Verifikation. 
  Es existieren verschiedene Repräsentationen von Nebenwirkungen, insbesondere 
  Monaden und algebraische Wirkungen.
  Um modulare Entwicklung von Programmen und Beweise über diese zu erlauben, ist
  die Modularität der Repräsentation entscheidend.
  Algebraische Wirkungen und Wirkungs-Behandler sind generell ein modularerer
  Ansatz als traditionelle Monaden-Transformer.
  Des weiteren sind algebraische Wirkungen eine bequeme Repräsentation für
  Nebenwirkungen in totalen Sprachen mit Dependent-Types.
  Zum Beispiel haben
  \textcite{DBLP:journals/programming/DylusCT19,DBLP:conf/haskell/ChristiansenDB19}
  diese Verwendet um ambiente Wirkungen, wie Haskells Parzialität, im
  Beweisassistenten Coq zu modellieren.

  Im traditionellen Ansatz können Operationen mit Sichtbarkeitsbereichen nur
  als Wirkungs-Behandler implementiert werden.
  \textcite{DBLP:conf/haskell/WuSH14} bemerkten fehlende Modularität im
  Vergleich zu Monaden-Transformern, wenn mehrere Wirkungen mit
  Sichtbarkeitsbereichen verwendet werden.
  Dies ist eine wichtige Einschränkung, da die meisten Wirkungen zugehörige
  Operationen mit Sichtbarkeitsbereichen haben.
  Um diese Limitation zu beheben, haben \textcite{DBLP:conf/haskell/WuSH14}
  Wirkungen mit Sichtbarkeitsbereichen vorgestellt.
  In Bezug auf Verifikation sind Wirkungen mit Sichtbarkeitsbereichen wichtig,
  da diese nötig sind, um komplexere ambiente Nebenwirkungen wie Sharing zu
  modellieren.

  In dieser Arbeit erkunden wir, wie gut sich drei Haskell-Implementierungen von
  Wirkungen mit Sichtbarkeitsbereichen durch
  \textcite{DBLP:conf/haskell/WuSH14,DBLP:conf/lics/PirogSWJ18}
  in die funktionale Programmiersprache mit Dependent-Types Agda übertragen
  lassen.
  Wir implementieren eine Bibliothek für Wirkungen mit Sichtbarkeitsbereichen.
  Diese erlaubt sowohl die Implementierung von wirkungsbehafteten Programmen,
  also auch deren Verifikation durch Agda als Beweisassistenten
  Um die Anwendbarkeit unserer Implementierung zu testen, implementieren wir
  Wirkungen, welche verwendet werden können um Currys~\cite{Hanus95curry}
  ambienten, nicht strikten Nichtdeterminismus zu modellieren.
  Unser Hauptbeispiel für eine Wirkung mit Sichtbarkeitsbereich ist Sharing von
  Nichtdeterminismus, nach \citeauthor{bunkenburg2019modeling}s Modellierung, um
  Currys Call-Time Choice Semantik zu simulieren.
  Um ambiente Wirkungen in Datenstrukturen zu modellieren, unterstützt die
  Implementierung tiefe Wirkungen.
\end{abstract}
\selectlanguage{english}

\newpage

\tableofcontents

\chapter{Introduction}

Computational effects such as state, nondeterminism and partiality are a
central concepts in functional programming and program verification
% , because usually programs interact with their environment.
A formal representation for computational effects is essential to allow
reasoning about programs using them.

Algebraic effects were first introduced by
\textcite{DBLP:journals/acs/PlotkinP03} as an alternative representation for
computational effects.
They represent a large class of effects using a set of operations and an
algebraic theory relating them.
Furthermore, they provide ways of constructing, reasoning and combining
effects~\cite{DBLP:journals/corr/PlotkinP13}.
Although, they can represent a large set of effects, not all operations from the
corresponding monad for the effect fit in the algebraic effect framework.
More precisely, they cannot represent non-algebraic operations, for example
operations with scopes, such as
\texttt{catch}~\cite{DBLP:journals/acs/PlotkinP03}.

Effect handlers were introduced by \textcite{DBLP:conf/esop/PlotkinP09} as a
generalization of exception handlers.
They allow modelling of non-algebraic operations and are used in conjunction
with algebraic effects~\cite{DBLP:journals/corr/PlotkinP13}.
Algebraic effects and handlers have a wide field of applications.
In functional programming they are used as a less boiler plate intensive
alternative to monad transformers.
% doppelt
They can be used as a theoretical foundation for reasoning about computational
effects.
Furthermore, they can be used to model semantics of programming languages.
Most functional programming languages allow a limited set of side
effects which are not represented in the type system of the language.
These effects are usually called \textit{ambient effects}.
For example, to allow easier development, Haskell programs can contain traces
and functions are allowed to be partial.
\textcite{DBLP:journals/programming/DylusCT19,DBLP:conf/haskell/ChristiansenDB19}
presented a modelling of ambient effect using free monads in Coq.
They showed that forms of algebraic effects and handlers are a convenient basis
for implementing effects in languages with dependent types and allow reasoning
about Haskell programs taking ambient effects into account.

\textcite{DBLP:conf/haskell/WuSH14} noticed a lack of modularity when
combining multiple effects with scoping operations because handlers delimit
scopes as well as define semantics.
They introduce two different solutions for the problem.
They fix the problem by introducing new operations to mark scopes and by
generalizing syntax to model scopes directly.
Scoped effects are usually used to implement effect systems in Haskell because
they provide a similar interface to normal monad transformers.
Furthermore, they are needed to simulate more complex semantics of programming
languages, such as sharing.


\section{Goals}

The goal of this thesis is the implementation of an effect library based on the
work by \textcite{DBLP:conf/haskell/WuSH14,DBLP:conf/lics/PirogSWJ18} in
the dependently typed, functional programming language
Agda~\cite{norell:thesis}.
Due to its stronger type system, Agda can be not only be used as a programming
language, but also as a proof assistant.
Therefore, Agda does not only allow writing programs with more expressive types,
but also allows verifying their properties.
To allow use as a proof assistant, Agda programs are subject to constraints not
present in other functional languages, such as Haskell.
All Agda programs have to be total (and therefore terminate), all data types
have to be strictly positive and all universe levels have to be consistent.
These restrictions prevent a direct translation of Haskell code to Agda and
therefore complicate porting of Haskell code to Agda.

This thesis explores multiple Haskell implementations of scoped effects and
studies how well these translate to Agda.
Furthermore we presents a set of standard solutions for common problems, arising
during the implementation of the Haskell approaches in Agda.

Agda's builtin theorem proving capabilities allow direct, machine checked
correctness proofs for the implemented library.
Alongside the implementation of scoped effects, the library contains proofs for
common properties of these effects.
Furthermore, the library itself can be used to simulate semantics of other
programming languages such as Curry's~\cite{Hanus95curry} call-time choice and
allow verification of equivalent programs written in these languages in Agda.
To model ambient effects in data structures correctly, the library has to be
able to represent deep effects i.e. data structures with effectful components.


\section{Structure}

Chapter \ref{chapter:preliminaries} contains short introductions to Agda and
dependent types, Curry and its call-time choice semantics (which are a central
example for a scoped effect throughout the thesis) and algebraic effects.

Chapter \ref{chapter:first-order} and \ref{chapter:higher-order} explore the two
approaches based on \citetitle{DBLP:conf/haskell/WuSH14} by
\textcite{DBLP:conf/haskell/WuSH14}.
The former of the two approaches was already partially explored by
\textcite{bunkenburg2019modeling} in Coq and can also be implemented
easily in Agda.
The latter of the two approaches tries to fix some inherent problems of the
first one, but sadly cannot fully be implemented in Agda due to some other
problems with universe consistency.

Chapter \ref{chapter:scoped-algebras} describes an implementation of a novel
representation for scoped effects by \textcite{DBLP:conf/lics/PirogSWJ18} which
also fixes the problems of the first approach, while avoiding the universe
inconsistencies of the second one.
Furthermore, the implementation explores some ideas to modularize the exemplary
Haskell implementation by \citeauthor{DBLP:conf/lics/PirogSWJ18}.

Chapter \ref{chapter:conclusion} summaries and compares the three approaches.
Furthermore, it gives an overview over related and possible future work.

\chapter{Preliminaries}
\label{chapter:preliminaries}
\input{Preliminaries}

\chapter{Implementing Scoped Effects in a First Order Setting}
\label{chapter:first-order}
\input{FirstOrder}

\chapter{Implementing Scoped Effects using Higher Order Syntax}
\label{chapter:higher-order}
\input{HigherOrder}

\chapter{Implementing Scoped Effects using Scoped Algebras}
\label{chapter:scoped-algebras}
\input{ScopedAlgebras}

\chapter{Conclusion}
\label{chapter:conclusion}

In this last chapter we summarize the work done in the thesis and discuss the
results as well as possible future work.

\section{Summary}

The main goal was to implement a library for scoped algebraic effects by
porting the two haskell implementations from
\citetitle{DBLP:conf/haskell/WuSH14} to Agda.

Of particular significance was to work under Agda's constraints for consistency,
which are needed for theorem proving.
These included defining only functions which Agda could prove terminate, defining
only strictly positive recursive types and work with a consistent universe
structure.
As a running example we tried implementing effects related to simulating Curry's
call-time choice semantics, namely nondeterminism and the sharing effect by
\textcite{bunkenburg2019modeling}.
In the chapters \ref{chapter:first-order}, \ref{chapter:higher-order} and
\ref{chapter:scoped-algebras} we discussed how three different approaches
translate to Agda.


\section{Results}

In Chapter \ref{chapter:first-order} we first implemented ordinary algebraic
effects in Agda.
The implementation only depends on the definition of the free monad, which
we implemented using containers, a well known approach.
To prove termination we used sized types, which proved effective because they
allowed the definition of \textcite{DBLP:conf/haskell/WuSH14} handlers for
explicit scope delimiters as well as avoided excessive inlining.
Furthermore, it was possible to implement the sharing handler by
\textcite{bunkenburg2019modeling} in Agda, making this approach a valid choice
for simulating call-time choice semantics.
A problem with this approach is that it inherently allows mismatched scope
delimiters, that is malformed syntax, which potentially complicates proofs.

In Chapter \ref{chapter:higher-order} we tried implementing effects using higher
order syntax, as described by \textcite{DBLP:conf/haskell/WuSH14}.
This approach turned out to be problematic, because it relies on existential
types to capture subcomputations.
Holding onto the idea of storing types directly, we showed that it is not
possible to implement deep effects in a setting with bounded universe levels.
Furthermore, we discussed potential alternative approaches, although non of them
seem promising.
The repository contains a sketch of an implementation with variable universe
levels, which showcases the practical difficulties.
Even thought this approach cannot model deep effects and therefore is not
expressive enough to model Curry's ambient nondeterminism in data structures, we
still presented the key points of a limited implementation of normal scoped
effects using higher order syntax.

Due to the deep-seated problem of the approach from Chapter
\ref{chapter:higher-order} we experimented with an novel representation by
\textcite{DBLP:conf/lics/PirogSWJ18}, which allowed us to remove the existential
type.
The naive Agda implementation of the monad \textcite{DBLP:conf/lics/PirogSWJ18}
proved to be problematic in terms of theorem proving.
We derived an alternative underlying implementation, based on the work by
\textcite{DBLP:journals/corr/abs-1806-05230}, which allowed the implementation
of complex recursive functions and proofs without the use of sized types.
The approach allows the implementation of deep effects and also can be used to
simulate sharing.
Furthermore, we experimented with a naive modularisation of scoped effects in
this setting.
The naive implementation seems promising in our limited, practical
setting and shows parallels to the higher-order approach. % of course, the monads are essentially equivalent


\section{Related Work and future Directions}

Different versions of effects and handlers are becoming more popular in the
functional programming community.
To still provide the usual structure and expressiveness of normal monad
transformers, most Haskell implementations (e.g. [polysemy], [fused-effects])
internally use higher order syntax to implement scoped effects.

In dependently typed language much less work has been done.
\textcite{DBLP:conf/icfp/Brady13} presented an implementation of resource
dependent effects in Idris.
Resource dependent effects are a version of algebraic effects in a dependently
typed setting.
An Agda version of the library by Norell exists.
\textcite{baanen2019algebraic} also implement algebraic effects in Agda.
His work focuses on program verification.


\subsection{Future Work}

For each of the three approach, the basic infrastructure as well as some
exemplary effects were presented.
Based on them, the corresponding library could be extended with more effects.
In case of the approaches from Chapter \ref{chapter:higher-order} and Chapter
\ref{chapter:scoped-algebras} a simplified syntax was used.
Some scoped operations, for example \texttt{pass} and \texttt{listen} for the
\texttt{Writer} effect, have fundamentally different shape, which cannot be
represented at the moment.
Implementing these operations requires generalizing the syntax.
For higher-order approach alternative container representations were mentioned.
For scoped algebras \textcite{DBLP:conf/lics/PirogSWJ18} also note a
generalization for more expressive syntax.

The modularisation of effect in the Chapter \ref{chapter:scoped-algebras} is
quite experimental.
Implementing more tests for the current set of handlers as well as adding more
effects with scoping operations to test the modularization should be the next
step.
Because we only explored an practical implementation of modular effects, we
should also examine the approach in a more formal setting.

Another area which could be explored further is reasoning about programs using
effects.
All proofs in the thesis either used a fixed effect stack or just hold if the
handlers for the effect was called last.
Most real program use multiple effects, which differ between functions.
Properties of effects may not hold if a non-orthogonal handler is called first.
Exploring this issue and defining structures for proving and using propositions
about effects in a modular setting, could be a useful addition to the library.

The approaches in this thesis are based on the work by
\textcite{DBLP:conf/haskell/WuSH14} and \textcite{DBLP:conf/lics/PirogSWJ18}.
Norell ported the Idris effect library by \textcite{DBLP:conf/icfp/Brady13} to
Agda.
The library does not allow scoped effects, but uses a fundamentally different
structure.
Extending this approach with scoped effects could provide an alternative to the
approaches described implemented in thesis.
Vice versa, one could try extending one of the approaches presented here with
recourse dependent effects.

\emergencystretch=2em
\printbibliography

\end{document}

