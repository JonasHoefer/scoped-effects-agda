\documentclass[10pt,a4paper,twoside]{report}

\usepackage[english]{babel}
\usepackage[a4paper,width=150mm,top=25mm,bottom=25mm]{geometry}

\usepackage{fontspec}
\usepackage[]{unicode-math}
\usepackage{wasysym}
\setmathfont{Latin Modern Math}
\setmathfont[range={\lBrace,\rBrace}]{XITSMath} % fallback for these two glyphe
\setsansfont{Latin Modern Sans}
% \setmainfont{Latin Modern Roman}
% \setmonofont{Fira Code}

\usepackage{newunicodechar}
\newunicodechar{ℕ}{\ensuremath{\mathnormal{\mathbb{N}}}}
\newunicodechar{∷}{\ensuremath{\mathnormal{::}}}
\newunicodechar{⊔}{\ensuremath{\mathnormal{\sqcup}}}
\newunicodechar{∀}{\ensuremath{\mathnormal{\forall}}}
\newunicodechar{ℓ}{\ensuremath{\mathnormal{\ell}}}
\newunicodechar{▷}{\ensuremath{\mathnormal{\rhd}}}
\newunicodechar{∈}{\ensuremath{\mathnormal{\in}}}
\newunicodechar{⦃}{\ensuremath{\mathnormal{\lBrace}}}
\newunicodechar{⦄}{\ensuremath{\mathnormal{\rBrace}}}
\newunicodechar{≅}{\ensuremath{\mathnormal{\cong}}}
\newunicodechar{⊕}{\ensuremath{\mathnormal{\oplus}}}
\newunicodechar{⊎}{\ensuremath{\mathnormal{\uplus}}}
\newunicodechar{λ}{\ensuremath{\mathnormal{\lambda}}}
\newunicodechar{∘}{\ensuremath{\mathnormal{\circ}}}
\newunicodechar{≡}{\ensuremath{\mathnormal{\equiv}}}
\newunicodechar{ˢ}{\textsuperscript{s}}
\newunicodechar{ˡ}{\textsuperscript{l}}
\newunicodechar{ʳ}{\textsuperscript{r}}
\newunicodechar{₁}{\textsubscript{1}}
\newunicodechar{₂}{\textsubscript{2}}

\usepackage{hyperref}
\usepackage[links]{agda}

\DeclareRobustCommand{\AgdaFormat}[2]{%
  \ifthenelse{\equal{#1}{>>=}}
  {\ensuremath{\mathbin{>\!\!\!>\mkern-6.7mu=}}}
  {\ifthenelse{\equal{#1}{\AgdaUnderscore{}>>=}}
  {\ensuremath{\AgdaUnderscore{}\mathbin{>\!\!\!>\mkern-6.7mu=}}}
  {\ifthenelse{\equal{#1}{>>=\AgdaUnderscore{}}}
  {\ensuremath{\mathbin{>\!\!\!>\mkern-6.7mu=}\AgdaUnderscore{}}}
  {\ifthenelse{\equal{#1}{\AgdaUnderscore{}>>=\AgdaUnderscore{}}}
  {\ensuremath{\AgdaUnderscore{}\mathbin{>\!\!\!>\mkern-6.7mu=}\AgdaUnderscore{}}}
  {\ifthenelse{\equal{#1}{>>}}
  {\ensuremath{\mathbin{>\!\!\!>}}}
  {\ifthenelse{\equal{#1}{\AgdaUnderscore{}>>\AgdaUnderscore{}}}
  {\AgdaUnderscore{}\ensuremath{\mathbin{>\!\!\!>}}\AgdaUnderscore{}}
  {#2}}}}}}}

\usepackage{graphicx}
\graphicspath{{images/}}

\usepackage{fancyhdr}
\pagestyle{fancy}
\fancyhf{}%
\fancyhead[LO,RE]{\thepage}
\renewcommand{\headrulewidth}{0.4pt}
\renewcommand{\footrulewidth}{0.0pt}

% Chapter pages
\fancypagestyle{plain}{%
  \fancyhf{}%
  \fancyhead[LO,RE]{\thepage}
  \renewcommand{\headrulewidth}{0.0pt}%
  \renewcommand{\footrulewidth}{0.0pt}%
}

\usepackage{minted}

\usepackage[style=alphabetic,backend=biber]{biblatex}
\addbibresource{thesis.bib}

\title{
  {\huge\textbf{Implementing a library for scoped algebraic effects in Agda}}\\
  {\large HTWK Leipzig}
}
\author{
  {Jonas Höfer}\\
  {Informatik}\\
  {69555}
}
\date{Juli 2020}

\begin{document}
\maketitle
\begin{abstract}
\end{abstract}

\tableofcontents

\chapter{Introduction}
\section{Goals}

\chapter{Preliminaries}
\section{Agda}

Agda is a functional language with dependent types.
The current version\footnote{\url{https://github.com/agda/agda}} was originally
developed by Ulf Norell \cite{norell:thesis}.
Due to its type system Agda can be used as a programming language and as a proof
assistant.

\begin{code}
data ℕ : Set where
  zero  : ℕ
  suc   : ℕ → ℕ
\end{code}
\begin{code}[hide]
{-# BUILTIN NATURAL ℕ #-}
\end{code}

\subsection{Dependent Types}

Dependent types are types which depend on values.
Types can have both other types and values as arguments.
Arguments on the left-hand side of the colon are called parameters and are the
same for all constructors of an algebraic data type.
Arguments on the right-hand side of the colon are called indices an can differ
for each constructor.

Consider the following definition of a vector.
The data type depends on a type \AgdaDatatype{A} and a value, a natural number.

\begin{code}
data Vec (A : Set) : ℕ → Set where
  _∷_  : {n : ℕ} → A → Vec A n → Vec A (suc n)
  []   : Vec A 0
\end{code}

The \AgdaInductiveConstructor{[]} constructor allows us to create an empty
vector of any type, but forces the index to be zero.
The \AgdaInductiveConstructor{\_∷\_} constructor appends an element to the front
of a vector of the same type, increasing the index in the process.
Only these two constructors can be used to construct vectors.
Therefore the index is always equal to the amount of elements stored in the
vector.

By encoding more information about the data in its type we can  [TODO: static
dynamic semantics / promote to type error / ...].
The following definition of \AgdaFunction{head} avoids error handling or
partiality by excluding the empty vector as a valid argument.

\begin{code}
head : ∀ {A n} → Vec A (suc n) → A
head (x ∷ _) = x
\end{code}

When pattern matching on the argument of \AgdaFunction{head} there is no case
for \AgdaInductiveConstructor{[]}.
The argument has type \AgdaDatatype{Vec A (suc n)} and
\AgdaInductiveConstructor{[]} has type \AgdaDatatype{Vec A 0}.
Those to types cannot be unified, because \AgdaInductiveConstructor{suc} and
\AgdaInductiveConstructor{zero} are different constructors of \AgdaDatatype{ℕ}.
Therefore the \AgdaInductiveConstructor{\_∷\_} case does not apply.

\subsection{Propositions as Types}

\cite{DBLP:journals/cacm/Wadler15}

\subsection{Strict Positivity}



\subsubsection{Container}

\subsection{Termination Checking}

The definition of non-terminating functions entails logical inconsistency.
Agda therefore only allows the definition terminating functions.
Due to the Undecidability of the halting problem Agda uses a heuristic termination checker.
The termination checker proofs termination by observing structural recursion.
Consider the following definitions of \AgdaDatatype{List} and \AgdaFunction{map}.

\begin{code}
data List (A : Set) : Set where
  _∷_  : A → List A → List A
  []   : List A

map : {A B : Set} → (A → B) → (List A → List B)
map f (x ∷ xs)  = f x ∷ map f xs
map f []        = []
\end{code}

The \AgdaInductiveConstructor{[]} case does not contain a recursive calls.
In the \AgdaInductiveConstructor{\_∷\_} case the recursive call to
\AgdaFunction{map} occurs on a structural smaller argument i.e.
\AgdaArgument{xs} is a subterm of the argument
\AgdaArgument{x}\;\AgdaInductiveConstructor{∷}\;\AgdaArgument{xs}.
Because elements of \AgdaDatatype{List A} are finite the function
\AgdaFunction{map} terminates for every argument.

\subsubsection{Sized Types}

\begin{code}
open import Agda.Builtin.Size public
  renaming ( SizeU to SizeUniv )  --  sort SizeUniv
  using    ( Size                 --  Size   : SizeUniv
           ; Size<_               --  Size<_ : Size → SizeUniv
           ; ↑_                   --  ↑_     : Size → Size
           ; _⊔ˢ_                 --  _⊔ˢ_   : Size → Size → Size
           ; ∞ )                  --  ∞      : Size

data Rose (A : Set) : Size → Set where
  rose : ∀ {i} → A → List (Rose A i) → Rose A (↑ i)

map-rose : {A B : Set} {i : Size} → (A → B) → (Rose A i → Rose B i)
map-rose f (rose x xs) = rose (f x) (map (map-rose f) xs)
\end{code}

\section{Curry and Call-Time-Choice}
let x = coin in x + x

\chapter{Algebraic Effects}
\input{FirstOrder}

\chapter{Higher Order}

\chapter{Conclusion}
\section{Summary}

\printbibliography

\end{document}
